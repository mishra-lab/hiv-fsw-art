%===================================================================================================
\subsection{Data Sources for Eswatini}\label{mod.par.data}
Major HIV data sources for Eswatini are summarized in Table~\ref{tab:data.esw},
and briefly described as follows.
Summary statistics were extracted from reports and publications in all cases,
except two FSW surveys \cite{Baral2014,EswKP2014},
for which individual-level data were obtained and analyzed directly in \sref{mod.par.fsw}.
\begin{table}
  \centering
  \caption{Main data sources for Eswatini}
  \label{tab:data.esw}
  \begin{tabular}{llllrl}
  \toprule
  Ref & ID & Dates\tn{a} & Population\tn{b} & N\tn{c} & HIV\tn{d} \\
  \midrule
  \cite{SDHS2006}    & DHS'06 & 07/06\,--\,02/07 & GP 15+    &  9,143 & P    \\
  \cite{SHIMS1}      & SHIMS1 & 12/10\,--\,06/11 & GP 18--49 & 18,169 & P, I \\
  \cite{SHIMS2}      & SHIMS2 & 08/16\,--\,03/17 & GP 15+    &  9,146 & P, I \\
  \cite{SHIMS3}\tn{e}& SHIMS3 & 05/21\,--\,11/21 & GP 15+    & 12,043 & P, I \\
  \cite{Baral2014}   & KP'11  & 09/11\,--\,10/11 & KP 15+    &    328 & P    \\
  \cite{EswKP2014}   & KP'14  & 09/14\,--\,01/15 & KP 18+    &    781 & ---  \\
  \cite{EswIBBS2022} & KP'21  & 10/20\,--\,01/21 & KP 18+    &    676 & P, I \\
  \bottomrule
\end{tabular}
\floatfoot{
  \tnt[a]{Baseline data collection (\textsc{mm/yy})};
  \tnt[b]{GP: general population; KP: key populations (female sex workers, men who have sex with men)};
  \tnt[c]{Respondents aged xx--49 who completed baseline survey};
  \tnt[d]{Estimates of HIV via blood test, P:~prevalence, I:~incidence};
  \tnt[e]{Preliminary findings only}.}


\end{table}
%---------------------------------------------------------------------------------------------------
\paragraph{General Population}
The 2006--07 Demographic and Health Survey (DHS) \cite{SDHS2006} was
the first nationally representative, household-based survey in Eswatini
covering numerous demographic and health topics.
The survey included dried blood spot HIV testing, covering 88.1\% of women and 81.1\% of men.
Adjusted HIV prevalence was stratified by sex, age, and other demographic factors, as well as
marital status and numbers of sexual partners in the past 12 months (p12m).
The survey also included data on sexual health and behaviour, including
condom use at last sex, STI symptoms in p12m, and HIV testing history.
The two SHIMS in 2010--11 \cite{SHIMS1} and 2016--17 \cite{SHIMS2} were conducted
with the aim of estimating population-level incidence before and after \emph{Soka Uncobe}.
Similar to the DHS, both SHIMS were nationally representative, household-based surveys;
however, SHIMS focused specifically on HIV variables, and additionally estimated
ART cascade steps and HIV incidence.
In SHIMS1 \cite{SHIMS1}, a large prospective 6-month cohort was used to estimate incidence
and validate recency testing \cite{Duong2012} as a cross-sectional measure of incidence, whereas
in SHIMS2 \cite{SHIMS2}, incidence was estimated via the validated recency test.
Compared to the DHS, participation rates were
lower in SHIMS1 (81.7\% and 65.0\% among women and men, for the baseline survey),
and similar in SHIMS2 (88.0\% and 78.5\%).
% TODO: SHIMS3
%---------------------------------------------------------------------------------------------------
\paragraph{Female Sex Workers}
The first behavioural surveillance survey among FSW in Eswatini
reached only 37 FSW during 2001--02 and did not include HIV testing \cite{EswIBBS2022}.
In 2011, a larger survey reached 328 FSW via respondent-driven sampling
and included HIV testing and detailed behavioural data \cite{Yam2013cond,Baral2014}.
This study found unadjusted HIV prevalence of 70.3\%,
highlighting a concentrated sub-epidemic among this key population
even within the high-prevalence Eswatini epidemic \cite{Baral2014}.
A follow-up study in 2014 aimed to
estimate FSW and MSM population sizes,
identify venues for HIV service delivery, and
provide additional data on service gaps \cite{EswKP2014};
this study used location-based snowball sampling \cite{Weir2005} to reach 781 FSW,
but did not include HIV testing.
Finally, a fourth survey in 2020--21 sought to estimate
FSW and MSM population sizes, HIV prevalence and incidence, prevalence of viral suppression,
as well as identify behavioural and structural factors associated with HIV \cite{EswIBBS2022};
the study recruited 676 FSW via respondent-driven sampling.