%===================================================================================================
\subsection{Calibration Targets}\label{mod.cal.targ}
The data sources for Eswatini calibration targets
are mainly the same as for Eswatini-specific parameters (see \sref{mod.par.data}).
We assumed that population-level surveys in
2006 (DHS)~\cite{SDHS2006},
2011 (SHIMS1)~\cite{Bicego2013,Justman2016},
2016 (SHIMS2)~\cite{SHIMS2}, and
2021 (SHIMS3)~\cite{SHIMS3}
reached FSW and their clients,
although respondents may not report selling or buying sex in the context of these surveys.
%---------------------------------------------------------------------------------------------------
\subsubsection{HIV Prevalence}\label{mod.cal.targ.prev}
% TODO: (*) clients prevalence ratio Carrasco2020 Table 2
\begin{table}
  \centering
  \caption{Estimates of HIV prevalence in Eswatini}
  \label{tab:targ.prev}
  % TODO: SHIMS3 N = 15+ not 15-49
\begin{tabular}{lcrccccll}
  \toprule
  Population\tn{a} & Year &      N & Raw \%  & Adj \%  &   (95\%~CI)  & Used & Ref & Notes \\
  \midrule
  Overall          & 2006 &  8,187 & 25.9 &  --- & (24.4,~27.3) & \yes & \cite{SDHS2006}   & \tn{b} \\
                   & 2011 & 18,172 & 32.1 & 28.0 & (27.0,~29.0) & \yes & \cite{Bicego2013} & \tn{cd} \\
                   & 2016 &  8,533 & 27.2 &  --- & (25.8,~28.7) & \yes & \cite{SHIMS2}     & \tn{b} \\
                   & 2021 & 12,043 & 23.7 &  --- & (22.6,~24.9) & \yes & \cite{SHIMS3}     & \tn{e} \\[1ex]
  Women Overall    & 2006 &  4,424 & 31.1 &  --- & (29.4,~32.9) & \yes & \cite{SDHS2006}   & \tn{b} \\
                   & 2011 &  9,843 & 38.8 & 34.2 & (33.0,~35.4) & \yes & \cite{Bicego2013} & \tn{cd} \\
                   & 2016 &  4,878 & 34.3 &  --- & (32.6,~36.0) & \yes & \cite{SHIMS2}     & \tn{b} \\
                   & 2021 &  6,985 & 31.6 &  --- & (29.8,~33.4) & \yes & \cite{SHIMS3}     & \tn{e} \\[1ex]
  Men Overall      & 2006 &  3,763 & 19.7 &  --- & (17.9,~21.4) & \yes & \cite{SDHS2006}   & \tn{b} \\
                   & 2011 &  8,329 & 24.1 & 20.7 & (19.6,~21.8) & \yes & \cite{Bicego2013} & \tn{cd} \\
                   & 2016 &  3,655 & 18.8 &  --- & (17.3,~20.4) & \yes & \cite{SHIMS2}     & \tn{b} \\
                   & 2021 &  5,058 & 15.6 &  --- & (14.3,~16.9) & \yes & \cite{SHIMS3}     & \tn{e} \\[1ex]
  LR Overall       & 2006 &  7,589 & 24.9 &  --- &      ---     & \no  & \cite{SDHS2006}   & \\
                   & 2011 & 16,145 & 31.9 &  --- &      ---     & \no  & \cite{Bicego2013} & \\
                   & 2016 &  7,887 & 32.2 &  --- &      ---     & \no  & \cite{SHIMS2}     & \\[1ex]
  Non-LR Overall   & 2006 &    579 & 38.3 &  --- &      ---     & \no  & \cite{SDHS2006}   & \\
                   & 2011 &  1,887 & 33.3 & 29.0 & (25.9,~32.2) & \no  & \cite{Bicego2013} & \tn{cd} \\
                   & 2016 &    914 & 28.7 &  --- & (25.8,~31.7) & \no  & \cite{SHIMS2}     & \tn{g}  \\[1ex]
  LR Women         & 2006 &  4,346 & 30.7 & 26.8 & (22.7,~28.7) & \ast & \cite{SDHS2006}   & \tn{f}  \\
                   & 2011 &  9,843 & 38.2 & 30.8 & (28.9,~32.8) & \ast & \cite{Bicego2013} & \tn{cf} \\
                   & 2016 &  5,203 & 36.5 & 31.5 & (30.0,~33.1) & \ast & \cite{SHIMS2}     & \tn{f}  \\[1ex]
  Non-LR Women     & 2006 &     72 & 53.0 &  --- & (41.5,~64.3) & \ast & \cite{SDHS2006}   & \tn{g}  \\
                   & 2011 &    373 & 54.5 & 48.1 & (41.5,~54.8) & \ast & \cite{Bicego2013} & \tn{cd} \\
                   & 2016 &    263 & 45.3 &  --- & (39.3,~51.3) & \ast & \cite{SHIMS2}     & \tn{g}  \\[1ex]
  LR Men           & 2006 &  3,243 & 17.1 & 14.1 &  (6.5,~16.7) & \ast & \cite{SDHS2006}   & \tn{f}  \\
                   & 2011 &  6,733 & 23.2 & 19.0 & (18.0,~20.1) & \ast & \cite{Bicego2013} & \tn{cf} \\
                   & 2016 &  2,684 & 25.1 & 16.9 & (15.7,~18.1) & \ast & \cite{SHIMS2}     & \tn{f}  \\[1ex]
  Non-LR Men       & 2006 &    506 & 36.1 &  --- & (32.0,~40.3) & \ast & \cite{SDHS2006}   & \tn{g}  \\
                   & 2011 &  1,515 & 28.1 & 24.1 & (21.4,~26.9) & \ast & \cite{Bicego2013} & \tn{cd} \\
                   & 2016 &    651 & 22.8 &  --- & (19.7,~26.1) & \ast & \cite{SHIMS2}     & \tn{g} \\[1ex]
  FSW Overall      & 2011 &    328 & 70.3 & 60.5 & (52.1,~69.0) & \yes & \cite{Baral2014}  & \tn{h} \\
                   & 2014 &    781 & 37.8 &  --- &      ---     & \no  & \cite{EswKP2014}  & \tn{i} \\
                   & 2021 &    676 & 60.8 & 58.8 & (53.9,~63.6) & \yes & \cite{Baral2014}  & \tn{h} \\
  \bottomrule
\end{tabular}
\floatfoot{%
  \tnt[a]{LR: lower risk, reporting 0-1 partners p6m;
    Non-LR: lower risk, reporting 2+ partners p6m;
    FSW: female sex worker};
  \tnt[b]{95\%~CI as reported from sampling adjustment};
  \tnt[c]{adjusted from ages 18--49 to 15--49 (see \sref{mod.cal.targ.prev})};
  \tnt[d]{95\%~CI expanded via inferred sampling adjustment};
  \tnt[e]{N for survey overall};
  \tnt[f]{adjusted for biased reporting of risk behaviours
    (see \sref{mod.par.wp} and \sref{mod.cal.targ.prev})};
  \tnt[g]{95\%~CI inferred from N};
  \tnt[h]{RDS-adjusted};
  \tnt[i]{self-reported};
  \ast used within prevalence ratio only;
  all estimates used the BAB distribution.}

\end{table}
Table~\ref{tab:targ.prev} summarizes the available HIV prevalence data for Eswatini.
Uncertainty around each estimate was modelled using a BAB distribution.
We made several adjustments to these estimates as described below.
%---------------------------------------------------------------------------------------------------
\paragraph{Sampling Error}
Population-level HIV prevalence estimates in 2006 and 2016 included
expanded 95\%~CI (\vs standard binomial 95\%~CI)
due to sampling error for women, men, and the population overall
(Table~B.2 in \cite{SDHS2006} and Table~C.2 in \cite{SHIMS2}).
This expanded 95\%~CI corresponds to a reduction in effective $N$ \vs the sample $N$
for the binomial distribution, by a factor of 41--75\%.
We applied this factor to equivalently expand the estimated 95\%~CI for
the corresponding lower risk and non-lower risk women, men, and population overall in 2006 and 2016,
and also for all 2011 HIV prevalence estimates \cite{Bicego2013}.
%---------------------------------------------------------------------------------------------------
\paragraph{Biased Partner Number Reporting}
As discussed in \sref{mod.par.wp}, we assumed that
the proportion of the population reporting 0--1 sexual partners p6m
(``lower risk'') is overestimated,
and the proportion reporting 2+ (``non-lower risk'') is underestimated.
While overall HIV prevalence estimates would not be affected by this reporting bias,
HIV prevalence among the lower risk group would be overestimated.
To correct this overestimate, we further assumed that
HIV prevalence among ``observed'' non-lower risk (had 2+ partners p6m, reported 2+)
was representative of HIV prevalence among ``unobserved'' non-lower risk (had 2+, reported 0--1).
Thus, HIV prevalence among the ``true'' lower risk (had 0--1, reported 0--1) can be estimated as:
\begin{equation}
  H_{01} = \frac{H - H_{2+}W'_{2+}}{W'_{01}}
\end{equation}
where $H$ denotes HIV prevalence,
and $W'$ denotes the adjusted proportions calculated in \sref{mod.par.wp}.
%---------------------------------------------------------------------------------------------------
\paragraph{Age Range}
The model aims to capture the Swati population aged 15--49.
While the 2006, 2016, and 2021 surveys provide data for ages 15--49,
the 2011 survey was limted to ages 18--49.
Since HIV prevalence is much lower among ages 15--17,
the 2011 estimates would be biased high.
We therefore adjusted all 2011 HIV prevalence estimates in as follows.
Drawing on age-stratified data in 2006 \cite{SDHS2006} and 2011 \cite{Bicego2013},
we assumed that HIV prevalence among ages 15--17 was
5\% among women, 2\% among men, and 3.5\% overall.
Next, we estimated the fraction of women aged 15--17 among all women aged 15--49 (13.5\%),
and likewise for men (15.4\%) and overall (14.4\%) \cite{DataBank}.
We then estimated HIV prevalence among women, men, and overall for ages 15--49
using a weighted average of the 15--17 and 18--49 estimates.
Finally, we computed the resulting relative reduction in HIV prevalence for women overall,
and applied this reduction equally to the HIV prevalence estimates for
lower risk and non-lower risk women, and likewise for men and the population overall.
\begin{table}
  \centering
  \caption{Estimated HIV prevalence ratios in Eswatini}
  \label{tab:targ.pr}
  \begin{tabular}{llccccll}
  \toprule
  Numerator\tn{a} & Denominator\tn{a} & Year & Ratio & (95\%~CI) & Used & Ref & Notes \\
  \midrule
  Non-LR Women & LR Women      & 2006 & 2.02 & (1.84,~2.34) & \yes & \cite{SDHS2006}   & \tn{b} \\
               &               & 2011 & 1.54 & (1.47,~1.66) & \yes & \cite{Bicego2013} & \tn{b} \\
               &               & 2016 & 1.42 & (1.37,~1.51) & \yes & \cite{SHIMS2}     & \tn{b} \\[1ex]
  Non-LR Men   & LR Men        & 2006 & 2.57 & (2.16,~5.28) & \yes & \cite{SDHS2006}   & \tn{b} \\
               &               & 2011 & 1.24 & (1.20,~1.34) & \yes & \cite{Bicego2013} & \tn{b} \\
               &               & 2016 & 1.32 & (1.26,~1.45) & \yes & \cite{SHIMS2}     & \tn{b} \\[1ex]
  FSW Overall  & Women Overall & 2011 & 2.16 & (1.87,~2.50) & \yes & \cite{Baral2014,Bicego2013} & \tn{b} \\
               &               & 2021 & 1.86 & (1.68,~2.06) & \yes & \cite{EswIBBS2022,SHIMS3} & \tn{b} \\[1ex]
  HR FSW       & LR FSW        & 2011 & 1.05 & (0.85,~1.27) & \no  & \cite{Baral2014}  & \tn{c} \\
  \bottomrule
\end{tabular}
\floatfoot{%
  \tnt[a]{LR: lower risk, reporting 0-1 partners p6m;
    Non-LR: lower risk, reporting 2+ partners p6m;
    FSW: female sex worker;
    HR/LR FSW: higher/lower risk FSW, as defined in \sref{mod.par.fsw}};
   \tnt[b]{mean and 95\%~CI estimated via Monte Carlo sampling};
   \tnt[c]{per analysis in \sref{mod.par.fsw}};
   see Table~\ref{tab:targ.prev} for more notes on data sources and adjustments.}

\end{table}
\par
The raw (unadjusted) estimates suggest that HIV prevalence strongly peaked between 2006 and 2016.
After adjustment for respondent ages, 2011 estimates remained highest,
but the magnitude of differences with 2006 and 2016 was reduced substantially.
The largest reduction in HIV prevalence via adjustment
was among lower risk women in 2011: from 38.2\% to 30.8\%, due to the modelled
``addition'' of women/girls aged 15--17 (lower HIV prevalence), and
``subtraction'' of women with 2+ partners p6m (higher HIV prevalence).
\paragraph{Prevalence Ratios}
Since risk heterogeneity is a key determinant of epidemic dynamics,
it is important to capture HIV prevalence ratios across risk groups.
For this objective, directly specifying prevalence ratio targets is more efficient than
using independent prevalence targets for lower risk and non-lower risk.
Based on the available data, we defined the prevalence ratio targets in Table~\ref{tab:targ.pr}.
%---------------------------------------------------------------------------------------------------
\subsubsection{HIV Incidence}\label{mod.cal.targ.inc}
Population-level HIV incidence was first measured in the 2011
Swaziland HIV Incidence Measurement Survey (SHIMS)
via 6-month cohort (gold standard) \cite{SHIMS1,Justman2016},
in which 145 seroconversions were observed among 11,232 re-tested (LTFU was 5.6\%).
SHIMS2 and SHIMS3 in 2016--17 and 2021 used
the HIV-1 Limiting Antigen Enzyme Immunoassay (LAg~EIA) ``recency test'',
which detects infections acquired within the past 141 days, 95\%CI: (119,~160) \cite{Duong2012};
this LAg~EIA incidence measure was validated during SHIMS1 \cite{SHIMS1}.
Recency testing was also recently integrated into Eswatini standard of care \cite{EswCOP21}.
\par
\begin{table}
  \centering
  \caption{Estimates of HIV incidence in Eswatini}
  \label{tab:targ.inc}
  \begin{tabular}{lcrccccll}
  \toprule
  Population\tn{a} & Year &      N & Raw \%  & Adj \%  &   (95\%~CI)  & Used & Ref & Notes \\
  \midrule
  Overall          & 2016 &  9,476 &  1.48 &  --- & (0.96,~1.99)  & \yes & \cite{SHIMS2}      & \tn{bc} \\
                   & 2021 & 12,043 &  0.77 &  --- & (0.39,~1.15)  & \yes & \cite{SHIMS3}      & \tn{d} \\[1ex]
  Women Overall    & 2011 &  5,486 &  3.1  & 2.94 & (2.52,~3.47)  & \yes & \cite{Justman2016} & \tn{ef} \\
                   & 2016 &  5,227 &  1.99 &  --- & (1.16,~2.80)  & \yes & \cite{SHIMS2}      & \tn{bc} \\
                   & 2021 &  6,985 &  1.45 &  --- & (0.69,~2.20)  & \yes & \cite{SHIMS3}      & \tn{d} \\[1ex]
  Men Overall      & 2011 &  5,746 &  1.7  & 1.50 & (1.16,~1.84)  & \yes & \cite{Justman2016} & \tn{ef} \\
                   & 2016 &  4,249 &  0.99 &  --- & (0.39,~1.59)  & \yes & \cite{SHIMS2}      & \tn{bc} \\
                   & 2021 &  5,058 &  0.20 &  --- & (0.00,~0.48)  & \yes & \cite{SHIMS3}      & \tn{d} \\[1ex]
  LR Women         & 2011 &  4,924 &  3.21 & 1.58 & (0.40,~2.24)  & \ast & \cite{Justman2016} & \tn{efg} \\
  Non-LR Women     & 2011 &     93 & 10.10 & 9.62 & (4.76,~18.29) & \ast & \cite{Justman2016} & \tn{ef} \\
  LR Men           & 2011 &  3,855 &  1.64 & 0.76 & (0.01,~1.17)  & \ast & \cite{Justman2016} & \tn{efg} \\
  Non-LR Men       & 2011 &    874 &  3.87 & 3.42 & (2.21,~4.94)  & \ast & \cite{Justman2016} & \tn{ef} \\[1ex]
  FSW Overall      & 2021 &    676 & 11.71 &  --- & (8.31,~16.92) & \yes & \cite{EswIBBS2022} & \tn{b} \\
  \bottomrule
\end{tabular}
\floatfoot{%
  \tnt[a]{LR: lower risk, reporting 0-1 partners p6m;
    Non-LR: lower risk, reporting 2+ partners p6m;
    FSW: female sex worker};
  \tnt[b]{via HIV-1 Limiting Antigen recency testing};
  \tnt[c]{95\%~CI as reported from sampling adjustment};
  \tnt[d]{N for survey overall};
  \tnt[e]{via 6 month cohort (94.4\% follow-up)};
  \tnt[f]{adjusted from ages 18--49 to 15--49 (see \sref{mod.cal.targ.prev})};
  \tnt[g]{adjusted for biased reporting of risk behaviours
    (see \sref{mod.par.wp} and \sref{mod.cal.targ.prev})};
  \ast used within incidence ratio only;
  all estimates used the skew normal distribution.}

\end{table}
Table~\ref{tab:targ.inc} summarizes the available HIV incidence data for Eswatini.
Uncertainty around each estimate was modelled using a skewnormal or inverse gaussian distribution.
As with prevalence, the 2011 estimates were adjusted for the missing 15--17 age range,
this time assuming 2\% and 0.4\% annual incidence
among women and men aged 15--17, respectively
(extrapolating from age-stratified incidence estimates from \cite{Justman2016}).
The 2011 estimates for lower risk women and men were also adjusted
for biased partner number reporting using the same approach as for HIV prevalence.
Two incidence ratios were also defined (Table~\ref{tab:targ.ir}).
\par
No study of FSW in Eswatini estimated incidence directly,
but \cite{EswIBBS2022} reported that 30 of 676 prevalent HIV infections among FSW
were identified as recent via LAg~EIA per national guidelines \cite{SHIMS2,EswCOP21}.
Using \eqref{eq:exp.decay} with $\rho = 30/676 = 4.44\%$ and $T = 130$ days,
we computed an incidence rate of $\lambda = 11.7\%$ per year.
We further estimated uncertainty for this rate by combining
the 95\%~CI from $\rho \sim \opname{Binom}(p = 4.44\%, N = 676)$ and $T \in (118,~140)$,
yielding 95\%~CI for $\lambda$ of (8.3,~16.9).
\begin{table}
  \centering
  \caption{Estimated HIV incidence ratios in Eswatini}
  \label{tab:targ.ir}
  \begin{tabular}{llccccll}
  \toprule
  Numerator\tn{a} & Denominator\tn{a} & Year & Ratio & (95\%~CI) & Used & Ref & Notes \\
  \midrule
  Non-LR Women & LR Women & 2011 & 5.74 & (2.47,~22.26) & \yes & \cite{Bicego2013} & \tn{b} \\[1ex]
  Non-LR Men   & LR Men   & 2011 & 4.16 & (1.69,~23.09) & \yes & \cite{Bicego2013} & \tn{b} \\
  \bottomrule
\end{tabular}
\floatfoot{%
  \tnt[a]{LR: lower risk, reporting 0-1 partners p6m;
    Non-LR: lower risk, reporting 2+ partners p6m;
    FSW: female sex worker};
  \tnt[b]{mean and 95\%~CI estimated via Monte Carlo sampling};
  see Table~\ref{tab:targ.inc} for more notes on data sources and adjustments.}

\end{table}
%---------------------------------------------------------------------------------------------------
\subsubsection{HIV Cascade of Care}\label{mod.cal.targ.cascade}
Table~\ref{tab:targ.cascade} summarizes the available data for the HIV cascade of care in Eswatini,
including estimates stratified by risk group where possible.
Both conditional (\eg on ART among diagnosed, ``90-90-90'')
and unconditional (\eg on ART among PLHIV, ``90-81-73'') cascade data were included,
which is redundant but may improve calibration quality.
Unlike HIV prevalence and incidence calibration targets, no adjustments were applied to these data.
A recent meta-analysis \cite{Soni2021} suggested substantial under-reporting of known HIV+ status,
including 9~(4,~15)\% among the population overall (10~studies),
and 32~(22,~44)\% among FSW specifically (2~studies).
However, data from SHIMS2 \cite{SHIMS2} suggested much lower under-reporting (2.2\%) in Eswatini.
% TODO: (?) consider adjustments ...
\begin{table}
  \centering
  \caption{Estimated HIV cascade of care in Eswatini}
  \label{tab:targ.cascade}
  \newcommand{\stephead}[1]{\multirow{3}{.1\linewidth}{#1}}
\begin{tabular}{llcrcccll}
  \toprule
  Step\tn{a} & Population\tn{a} & Year & N & \% & (95\%~CI) & Used & Ref & Notes \\
  \midrule
  \stephead{Diagnosed among PLHIV}
  & Overall          & 2011 & 5,807 & 62.6 & (61.4,~63.8) & \yes & \cite{SHIMS1T}    & \tn{bc} \\
  &                  & 2016 & 2,417 & 86.1 & (84.7,~87.6) & \yes & \cite{SHIMS2}     & \tn{e}  \\[1ex]
  & Women overall    & 2011 & 3,810 & 69.1 & (67.6,~70.6) & \yes & \cite{SHIMS1T}    & \tn{b}  \\
  &                  & 2016 & 1,690 & 90.2 & (88.6,~91.8) & \yes & \cite{SHIMS2}     & \tn{e}  \\[1ex]
  & Men overall      & 2011 & 1,997 & 50.1 & (47.9,~52.3) & \yes & \cite{SHIMS1T}    & \tn{b}  \\
  &                  & 2016 &   727 & 77.3 & (74.0,~80.6) & \yes & \cite{SHIMS2}     & \tn{e}  \\[1ex]
  & FSW              & 2011 &   313 & 74.1 & (61.7,~89.8) & \yes & \cite{EswR2P2013} & \tn{d}  \\
  &                  & 2021 &   411 & 88.3 & (85.1,~91.2) & \yes & \cite{EswIBBS2022}& \tn{bf}  \\[2ex]
  \stephead{On ART among Diagnosed}
  & Overall          & 2011 & 3,635 & 52.1 & (50.5,~53.7) & \yes & \cite{SHIMS1T}    & \tn{bcd} \\
  &                  & 2016 & 2,113 & 87.8 & (86.0,~89.6) & \yes & \cite{SHIMS2}     & \tn{e}   \\[1ex]
  & Women overall    & 2011 & 2,633 & 48.0 & (46.1,~49.9) & \yes & \cite{SHIMS1T}    & \tn{bd}  \\
  &                  & 2016 & 1,532 & 87.5 & (85.4,~89.6) & \yes & \cite{SHIMS2}     & \tn{e}   \\[1ex]
  & Men overall      & 2011 & 1,002 & 62.7 & (59.7,~65.7) & \yes & \cite{SHIMS1T}    & \tn{bd}  \\
  &                  & 2016 &   581 & 88.4 & (85.2,~91.6) & \yes & \cite{SHIMS2}     & \tn{e}   \\[1ex]
  & FSW              & 2011 &   174 & 36.9 & (30.1,~44.2) & \yes & \cite{EswR2P2013} & \\
  &                  & 2021 &   363 & 97.5 & (95.7,~98.9) & \yes & \cite{EswIBBS2022}& \tn{bf} \\[2ex]
  \stephead{On ART among PLHIV}
  & Overall          & 2011 & 5,807 & 31.9 & (30.7,~33.1) & \yes & \cite{SHIMS1T}    & \tn{bc} \\
  &                  & 2016 & 2,417 & 75.6 & (73.6,~77.5) & \yes & \cite{SHIMS2}     & \tn{e} \\[1ex]
  & Women overall    & 2011 & 3,810 & 33.2 & (31.7,~34.7) & \yes & \cite{SHIMS1T}    & \tn{b} \\
  &                  & 2016 & 1,690 & 78.9 & (76.8,~81.1) & \yes & \cite{SHIMS2}     & \tn{e} \\[1ex]
  & Men overall      & 2011 & 1,997 & 31.4 & (29.4,~33.4) & \yes & \cite{SHIMS1T}    & \tn{b} \\
  &                  & 2016 &   727 & 68.3 & (64.7,~72.0) & \yes & \cite{SHIMS2}     & \tn{e} \\[1ex]
  & FSW              & 2011 &   313 & 27.4 & (20.9,~35.7) & \yes & \cite{EswR2P2013} & \tn{d} \\
  &                  & 2021 &   411 & 86.1 & (82.6,~89.3) & \yes & \cite{EswIBBS2022}& \tn{bf} \\[2ex]
  \stephead{VLS among On ART}
  & Overall          & 2016 & 1,858 & 90.3 & (89.0,~91.6) & \yes & \cite{SHIMS2}  & \tn{e} \\
  & Women overall    & 2016 & 1,342 & 91.4 & (89.9,~92.8) & \yes & \cite{SHIMS2}  & \tn{e} \\
  & Men overall      & 2016 &   516 & 87.6 & (84.4,~90.9) & \yes & \cite{SHIMS2}  & \tn{e} \\[2ex]
  \stephead{VLS among PLHIV}
  & Overall          & 2016 & 2,417 & 68.2 & (66.1,~70.4) & \yes & \cite{SHIMS2}  & \tn{e} \\
  &                  & 2021 & 2,854 & 86.6 & (85.0,~88.1) & \yes & \cite{SHIMS3}  & \tn{g} \\
  & Women overall    & 2016 & 1,690 & 72.1 & (69.7,~74.5) & \yes & \cite{SHIMS2}  & \tn{e} \\
  &                  & 2021 & 2,207 & 88.6 & (87.0,~90.2) & \yes & \cite{SHIMS3}  & \tn{g} \\
  & Men overall      & 2016 &   727 & 59.9 & (56.1,~63.7) & \yes & \cite{SHIMS2}  & \tn{e} \\
  &                  & 2021 &   789 & 82.4 & (79.3,~85.5) & \yes & \cite{SHIMS3}  & \tn{g} \\
  \bottomrule
\end{tabular}
\floatfoot{%
  \tnt[a]{PLHIV: people living with HIV;
    ART: antiretroviral therapy;
    VLS: HIV viral load suppressed, defined as $\le$\,1000 RNA copies/mL in \cite{SHIMS2};
    FSW: female sex worker};
  \tnt[b]{95\%~CI inferred from N};
  \tnt[c]{estimated from combining women \& men};
  \tnt[d]{estimated from conditional steps, with 95\%~CI via simulation};
  \tnt[e]{95\%~CI as reported from sampling adjustment};
  \tnt[f]{not RDS-adjusted};
  \tnt[g]{N estimated from HIV prevalence};
  \cite{SDHS2006} did not provide any appropriate cascade data.}
% TODO: (?) comment on SHIMS2 Table 8.4.A
% concordance of self-reported treatment status vs presence of antiretrovirals

\end{table}
\par
