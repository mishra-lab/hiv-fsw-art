%===================================================================================================
\subsection{Risk Differences Within Sex Work}\label{mod.par.fsw}
Compartmental HIV transmission models which include FSW
have rarely sub-stratified FSW (besides age) \cite{Knight2022sr,Cremin2017,Low2015,Shannon2015},
such as to reflect differential HIV risk or distinct typologies of sex work
\cite{Blanchard2008,Scorgie2012};
yet such heterogeneities may influence transmission dynamics.
Our model structure (Figure~\ref{fig:model.risk})
was designed to capture \emph{within}-FSW risk heterogeneity.
The objective of the following analysis was therefore to parameterize higher \vs lower risk sex work.
For this analysis, we used individual-level data from
two biobehavioural surveys among Swati FSW in
2011 \cite{Baral2014,Yam2013cond} (N = 325) and 2014 \cite{EswKP2014} (N = 781).
More details about each study are given in \sref{mod.par.data}.
\par
Based on community input,%
\footnote{Personal communication: Lungile Khumalo, \emph{Voice of Our Voices}, Eswatini}
we conceptualized risk differences within sex work as transient periods of higher \vs lower risk,
rather than distinct types of sex worker.
As such, we modelled rapid turnover between higher and lower risk FSW (see \sref{mod.par.turn}),
and distinguished these states via the total numbers of clients in p1m.
Specifically, we stratified survey respondents into the top 20\% / bottom 80\%
in total numbers of new and regular clients reported for p1m.
We then summarized key variables within the two strata (Table~\ref{tab:fsw.ratios})
and estimated the ratio of means per \cite{Hauschke1999}.
We repeated this analysis using 2011, 2014, and combined datasets.
\par
Years selling sex, non-paying partners, condom use, anal sex, and HIV status (2011 data only)
did not differ substantially between strata,
while reported numbers of clients and STI symptoms did.
We sampled reported numbers of clients in p1m from gamma distributions with $\alpha = 25$,
reflecting an assumption of mean / 5 = standard deviation;
means were specified as:
14 and 21 for new and regular clients in higher risk sex work, and
3.5 and 6 for new and regular clients in lower risk sex work.
We further use data in Table~\ref{tab:fsw.ratios} regarding:
partners and clients in \sref{mod.par.pnum},
STI symptoms in \sref{mod.par.beta.gud},
condom use in \sref{mod.par.tm.condom},
anal sex in \sref{mod.par.fsex},
years selling sex in \sref{mod.par.turn}.
% TODO: (?) anal sex declining?
\begin{table}
  \centering
  \caption{Ratios of variables among higher \vs lower risk FSW in Eswatini}
  \label{tab:fsw.ratios}
  \begin{tabular}{llRcRcRc}
  \toprule
  & & \multicolumn{2}{c}{Higher} & \multicolumn{2}{c}{Lower} & \multicolumn{2}{c}{Ratio} \\
  \cmidrule(rl){3-4}\cmidrule(rl){5-6}\cmidrule(rl){7-8}
  Year & Variable & mean & (range) & mean & (range) & mean & (95\%~CI) \\
  \midrule 2011
  & Age                                  & 24.5 &  (17, 41)  & 26.6 & (16, 49) & 0.92 & (0.87, 0.98) \\
  & Years selling sex                    & 4.82 &  (0, 18)   & 5.76 & (0, 30)  & 0.84 & (0.63, 1.06) \\
  & Non-paying partners p1m              & 1.32 &   (0, 5)   & 1.45 &  (0, 6)  & 0.91 & (0.71, 1.12) \\
  & New clients p1m                      & 18.9 &  (0, 60+)  & 3.87 & (0, 15)  & 4.87 & (3.55, 6.27) \\
  & Regular clients p1m                  & 23.9 &  (3, 60+)  & 5.71 & (0, 20)  & 4.19 & (3.40, 5.04) \\
  & Non-paying partner condom use\tn{a}  & 0.51 &    ---     & 0.49 &   ---    & 1.04 & (0.74, 1.39) \\
  & New client condom use\tn{a}          & 0.93 &    ---     & 0.87 &   ---    & 1.07 & (0.98, 1.17) \\
  & Regular client condom use\tn{a}      & 0.78 &    ---     & 0.83 &   ---    & 0.94 & (0.80, 1.08) \\
  & Any anal sex p1m\tn{a}               & 0.37 &    ---     & 0.47 &   ---    & 0.79 & (0.50, 1.11) \\
  & Any STI symptoms p12m\tn{a}          & 0.60 &    ---     & 0.48 &   ---    & 1.24 & (0.95, 1.57) \\
  & HIV status\tn{ab}                    & 0.72 &    ---     & 0.70 &   ---    & 1.03 & (0.85, 1.22) \\
  \midrule 2014
  & Age                                  & 27.1 &  (18, 44)  & 27.6 & (18, 50) & 0.98 & (0.95, 1.01) \\
  & Years selling sex                    & 6.12 &  (0, 22)   & 6.44 & (1, 26)  & 0.95 & (0.83, 1.08) \\
  & Non-paying partners p1m              & 1.43 &  (0, 17)   & 1.15 & (0, 10)  & 1.25 & (0.91, 1.61) \\
  & New clients p1m                      & 12.2 &  (0, 60+)  & 3.35 & (0, 16)  & 3.65 & (3.10, 4.23) \\
  & Regular clients p1m                  & 20.0 &  (0, 60+)  & 6.21 & (0, 20)  & 3.22 & (2.89, 3.58) \\
  & Non-paying partner condom use\tn{a}  & 0.71 &    ---     & 0.83 &   ---    & 0.85 & (0.73, 0.98) \\
  & New client condom use\tn{a}          & 0.89 &    ---     & 0.89 &   ---    & 1.00 & (0.93, 1.07) \\
  & Regular client condom use\tn{a}      & 0.82 &    ---     & 0.87 &   ---    & 0.94 & (0.86, 1.02) \\
  & Any anal sex p1m\tn{a}               & 0.13 &    ---     & 0.08 &   ---    & 1.69 & (0.93, 2.70) \\
  & Any STI symptoms p12m\tn{a}          & 0.30 &    ---     & 0.22 &   ---    & 1.34 & (0.98, 1.77) \\
  \midrule Both
  & Age                                  & 26.2 &  (17, 44)  & 27.3 & (16, 50) & 0.96 & (0.93, 0.99) \\
  & Years selling sex                    & 5.63 &  (0, 20)   & 6.27 & (0, 30)  & 0.90 & (0.80, 1.01) \\
  & Non-paying partners p1m              & 1.39 &  (0, 17)   & 1.25 & (0, 10)  & 1.12 & (0.90, 1.35) \\
  & New clients p1m                      & 14.2 &  (0, 60+)  & 3.53 & (0, 20)  & 4.03 & (3.45, 4.63) \\
  & Regular clients p1m                  & 21.2 &  (0, 60+)  & 6.05 & (0, 20)  & 3.51 & (3.18, 3.85) \\
  & Non-paying partner condom use\tn{a}  & 0.62 &    ---     & 0.71 &   ---    & 0.88 & (0.76, 1.01) \\
  & New client condom use\tn{a}          & 0.90 &    ---     & 0.88 &   ---    & 1.02 & (0.96, 1.08) \\
  & Regular client condom use\tn{a}      & 0.80 &    ---     & 0.86 &   ---    & 0.94 & (0.87, 1.01) \\
  & Any anal sex p1m\tn{a}               & 0.19 &    ---     & 0.19 &   ---    & 1.01 & (0.70, 1.36) \\
  & Any STI symptoms p12m\tn{a}          & 0.40 &    ---     & 0.30 &   ---    & 1.33 & (1.08, 1.61) \\
  & HIV status\tn{ab}                    & 0.75 &    ---     & 0.69 &   ---    & 1.07 & (0.90, 1.27) \\
  \bottomrule
\end{tabular}
\floatfoot{
  Higher / Lower: top 20\% / bottom 80\% by total clients p1m;
  \tnt[a]{proportion of respondents};
  \tnt[b]{2011 data only (serologic HIV status)}.}

\end{table}
