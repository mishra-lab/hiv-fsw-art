\section{Introduction}\label{intro}
In compartmental models of infectious disease transmission,
the ``force of infection'' equation defines
the rate at which susceptible individuals acquire infection.
This equation therefore mechanistically integrates
various factors relevant to transmission.
In the simplest compartmental transmission models,
the force of infection $\lambda$ can be defined as:
\begin{equation}\label{eq:foi.simple}
  \lambda = C \beta \frac{I}{N}
\end{equation} where:
$C$ is the average contact rate per-person,
$\beta$ is the average probability of transmission per contact, and
$I/N$ is the current prevalence of infection.%
\footnote{\eqref{eq:foi.simple} assumes
  frequency-dependent rather than density-dependent transmission,
  which is almost always more appropriate for sexually-transmitted infections \cite{Begon2002}.}
If the population is stratified into multiple groups $i$,
infection is stratified into multiple states $h$, and
contacts are stratified into multiple types $p$,
then \eqref{eq:foi.simple} can be generalized~to:
\begin{equation}\label{eq:foi.strat}
  \lambda_i = \sum_{pi'h'} C_{pii'} \beta_{ph'} \frac{I_{i'h'}}{N_{i'}}
\end{equation}
where:
$C_{pii'}$ is the average rate of type-$p$ contacts per-person among group $i$ with group $i'$,
$\beta_{ph'}$ is the average probability of transmission per type-$p$ contact given infection stage $h'$, and
$I_{i'h'}/N_{i'}$ is the prevalence of infection state $h'$ among group $i'$.%
\footnote{\eqref{eq:foi.strat} further assumes that
  contact rate and mixing by infection state is random.}
The contact matrix $C_{pii'}$ is often specified to reflect
complex mixing patterns among risk groups,
which are key determinants of transmission dynamics \cite{??}
\par
The force of infection equation is further complicated by
repeated contacts with the same individuals, such as
in sexual partnerships, shared households, and other social relationships.
With repeated \vs random contacts, individuals who recently acquired or transmitted infection
will often continue to contact the same person, resulting in ``inert'' contacts
--- also called ``wasted'' or ``post-transmission'' contacts ---
and slower infection spread through the contact network \vs without inert contacts \cite{??}.
Models of sexually transmitted infections like HIV
--- in which contacts reflect individual sex acts ---
must therefore accurately capture dynamics of sexual partnerships \cite{Rao2021}.
Specifically, such models must include patterns of who partners with whom, and
some adjustment for inert sex acts (within partnerships where transmission has already occurred).
\par
During our previous review of compartmental models of HIV transmission \cite{Knight2022sr},
we noted several different approaches to capturing sexual partnership dynamics
within HIV force of infection equations.
That is, equations differed not only in which
risk groups, partnership types, health states, and/or interventions were modelled,
but also in which mathematical approximations of sexual partnership dynamics were used
(see also \cite{Rao2021}).
Previous work comparing different modelling frameworks
--- \ie compartmental \vs pair-formation \vs individual-based models ---
has shown that similar differences can influence key model outputs,
such as inferred parameter values and projected intervention impacts
\cite{Kretzschmar1998,Eames2002,Lloyd-Smith2004,Johnson2016mf}.
However, no study has examined the differences that we identified
among \emph{compartmental} model force of infection equations.
\par\pagebreak % TEMP
Therefore, regarding different approaches to modelling HIV transmission
via sexual partnerships in compartmental models, we sought to:
critically review assumptions and limitations of prior approaches (\sref{foi.prior}),
propose a new approach which overcomes several limitations of prior approaches (\sref{foi.prop}),
and compare key model outputs under prior \vs proposed approaches (\sref{foi.exp}).
