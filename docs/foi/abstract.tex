%SM: reall well-written abstract --> clear and easy to follow for a general scientific / modeling audience, which is hard to do with methodsy-abstracts :) 
In classic compartmental models of HIV and other sexually transmitted infections, %SM: am wondering if we should remove STI here, even though its true, just to keep the focus on HIV as the case-study throughout abstract. 
populations are stratified into homogeneous, memoryless states
such that timing of transmission within specific sexual partnerships cannot be tracked.
These models define the force of infection (incidence rate) via
mean rates of partnership change, multiplied by
cumulative probabilities of transmission per partnership or partnership-year.
Thus, partnerships are effectively modelled as instantaneous.
In this paper, we critically review %SM: nice!
the assumptions and limitations of this ``instantaneous partnerships'' approach,
including different variations thereof.
We then propose a new approach, the Effective Partnerships Adjustment (EPA),
which overcomes several of these limitations while retaining the compartmental nature of these transmission dynamic models. %SM: thinking ahead in abstract re: ABM...
EPA adds a new population stratification to track
individuals in partnerships where transmission has already occurred,
and then reduces these individuals' effective partnerships by one,
until they change partners. %SM: nice
Although pair-based compartmental models are another solution, unlike pair-based models, %SM: not 100% sure we need this first part, but added to provide some "intro" to the why we mentioned pair-based here - see what you think?
 EPA only adds one stratification per partnership type,
and can therefore easily handle high levels of partnership concurrency. %very nice
We implemented EPA and three instantaneous partnership variations
in an existing model of heterosexual HIV transmission in Eswatini,
and examined differences in model outputs
under equal and approach-specific (recalibrated) parameters.
We found that model outputs were similar between EPA and
one instantaneous approach that allowed partnership change rates $\ll$ 1, whereas
two instantaneous approaches that forced partnership change rates to be $\ge$ 1
severely overestimated transmission via long-term partnerships,
even after recalibrating parameters, relative to EPA.
%SM: how about a concluding statemtent like this instead? 
When modelling the transmission dynamics of HIV, the EPA approach offers an opportunity to reduce potential biases when 
drawing inference or generating predictions related to transmissions or interventions in medium to long-term sexual partnerships.

%Future modellers and holders should continue to interrogate
%how best to model HIV transmission via sexual partnerships.
