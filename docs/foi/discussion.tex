% TODO: (!) big re-order pars
\section{Discussion}\label{foi.disc}
Compartmental models of HIV transmission continue to guide the epidemic response
through projection of resource needs and analysis of intervention scenarios \cite{??}.
As such, accurate estimation of transmission rates along specific pathways
--- \ie to/from which risk groups and via which sexual partnership types --- remains essential.
Here we have critically reviewed key assumptions among existing approaches to
modelling HIV transmission via sexual partnerships in compartmental models.
We formalized distinctions between:
within- \vs between-partnership heterogeneity
when calculating the average probability of transmission per partnership;
risk per partnership \vs per partnership-year
when adjusting for inert sex acts; and
incidence rate \vs incidence proportion
when aggregating risk across multiple partnerships.
We also proposed a new approach: the \emph{Effective Partnerships Adjustment} (EPA),
which can overcome some of the key limitations of prior approaches
without the need to change modelling frameworks.
Finally, through model comparison experiments, we showed that
approaches based on risk per partnership-year (\vs partnership) can
substantially overestimate the relative contribution of
longer \vs shorter partnerships to overall transmission,
even after recalibrating the model under each approach.
\par % per partnership probabilities --------------------------------------------------------------
Models with instantaneous partnerships must compute
average probabilities of transmission per partnership.
The choice of averaging equation implies either
within-partnership heterogeneity or between-partnership heterogeneity
in the probability of transmission per sex act.
The correct choice is not always obvious, even for empirical data.
For example, survey questions like
\shortquote{Did you use a condom the last time you had sex?}
cannot distinguish between
50\% condom use in 100\% of partnerships \vs 100\% condom use in 50\% of partnerships.
While we have shown that these two cases can be relative by an inequality,
we only found notable differences (\eg~$>10$\%) when cumulative risk is large
--- \eg a high volume of sex and/or risk-increasing transmission modifiers.
Thus, the distinction between within- \vs between-partnership heterogeneity
may be of little consequence.
On the other hand, two major challenges remain for
calculating average probabilities of transmission per partnership.
First, relative risks associated with transmission modifiers
are typically quantified at the per act (\vs per partnership) level, using
exposure-stratified individual-level data \cite{Gray2001,Wawer2005,Boily2009}.
Yet, these relative risks have been directly applied to
average probabilities of transmission per partnership in several models.
This approach would then
underestimate the impact of risk-reducing modifiers (\eg condoms) and
overestimate the impact of risk-increasing modifiers (\eg genital ulcer disease).%
\footnote{Modifying the transmission probability via relative risk $R$ ---
  per-act: $B_a = (1 - {(1 - R\beta)}^A)$ \vs per-partnership: $B_p = R\,(1 - {(1 - \beta)}^A)$;
  thus: $B_a > B_p$ if $R < 1$, and $B_a < B_p$ if $R > 1$.}
Second, it remains unclear how \emph{dynamic} transmission modifiers
(\eg condom use, infection stage, )
should be modelled when aggregating sex acts across many years within longer partnerships.
\par % seroconcordance: accumulation & data -------------------------------------------------------
Transmission within sexual partnerships naturally generates seroconcordant-infected partnerships.
As shown in Figure~\ref{fig:fit.tdsc.foi}, seroconcordance therefore
accumulates within sexual partnerships during epidemic growth,
and can later decline as incidence declines.
This mechanism, captured by EPA but not instantaneous approaches,
decouples incidence from prevalence.
An earlier version of this work \cite{Knight2022smdm}
described this accumulation as \shortquote{partnership-level herd effects}, while
Eames and Keeling \cite{Eames2002} describe it as
\shortquote{correlation of infection statuses of neighboring individuals}.
There are two main implications of this seroconcordance perspective.
First, an alternate adjustment for inert sex acts in compartmental HIV models
could make use of empirical data on seroconcordant partnerships.
Such an adjustment should carefully consider potential sources of biases and time trends.
Second, efforts to quantify serosorting --- preferential selection of sexual partners
with matching (perceived) HIV serostatus \cite{Cassels2013} ---
may need to focus on new partnerships or longitudinal data \cite{Kim2020},
since it may not be possible to distinguish intentional seroconcordance
from transmission-driven seroconcordance in cross-sectional data \cite{Cassels2009}.
\par % prior work & 1-year ------------------------------------------------------------------------
Prior comparisons of instantaneous partnership models with
pair-based models \cite{Kretzschmar1998,Eames2002,Lloyd-Smith2004} and
a stochastic dynamic network model \cite{Johnson2016mf}
have shown that instantaneous partnerships
can overestimate the initial epidemic growth rate and equilibrium prevalence.
Such findings seem intuitive due to
the immediate risk of onward transmission after acquisition under instantaneous partnerships.
However, in \sref{foi.exp.model}, we showed how
epidemic dynamics under instantaneous partnerships strongly depend
on the partnership durations and change rates used.
That is, when durations were effectively capped at 1 year,
the adjustment for inert sex acts has little effect,
and modelled incidence was indeed overestimated relative to EPA.
By contrast, when full partnership durations were used,
modelled incidence was surprisingly similar to EPA.
No adjustments for inert sex acts were described in \cite{Eames2002,Lloyd-Smith2004}, and
the adjustments in \cite{Johnson2016mf} did not consider full partnership durations,%
\footnote{The adjustments in \cite{Johnson2016mf} considered
  1 month for main/spousal, 6 months for casual, and nothing for sex work partnerships;
  thus, repeat sex work contacts were not considered.}
making them closer to the partnership-year approaches.
\par % existing evidence / risk groups ------------------------------------------------------------
The results in \sref{foi.exp.mod.tpaf} suggest that
different force of infection approaches can influence model-estimated prevention priorities,
mainly when the available data for model calibration cannot inform
the relative contribution of specific transmission pathways.
These data should include, at minimum,
infection prevalence and/or incidence estimates for all modelled risk groups,
which seemed to align the contribution of sex work across approaches in our results.
However, the relative contribution of partnership types formed by multiple risk groups
--- \eg main/spousal and casual partnerships ---
may be much harder to identify using commonly available epidemiological data.
Thus, models underestimating inert sex acts are likely to systematically overestimate
the relative contribution of longer \vs shorter partnerships to overall transmission.
% TODO: (*) revise below
These results are corroborated by Johnson and Geffen \cite{Johnson2016mf}, who concluded:
\shortquote{Frequency-dependent models are likely to
  underestimate the importance of interventions that are targeted at high-risk groups,
  while overestimating the impact of interventions targeted at low-risk groups.}
While higher risk groups typically have more shorter partnerships,
higher risk is not necessarily synonymous with shorter partnerships.
Indeed, the Eswatini model includes both
shorter (casual) partnerships among the lowest risk groups and
longer (regular sex work) partnerships among the highest risk groups.
This distinction between risk groups and partnership types
is therefore important to keep in mind
when interpreting these results and their potential implications.
\par % alaternate modelling frameworks ------------------------------------------------------------
The 2021 review by \citet{Rao2021} summarizes
modelling frameworks that have been used to
approximate partnership dynamics for modelling sexually transmitted infections.%
\footnote{See also \sref{sr.foi.alt} and Appendix~1 of \cite{Johnson2016mf}.}
Besides pair-based models, the review does not identify another approach
which can extend compartmental models beyond instantaneous partnerships.
However, several hybrid models have been developed for HIV \cite{Xiridou2003,Powers2011ahi}
wherein long-term pair are explicitly modelled,
but additional partnerships are modelled as instantaneous.
When long-term partnership concurrency is low, such hybrid approaches
may offer substantial improvements over fully instantaneous partnerships
\cite{Kretzschmar1998,Eames2002,Lloyd-Smith2004}.
However, the high number of repeat clients reported by Swati FSW (\sref{mod.par.fsw})
reflects precisely the high level of concurrency
which is difficult to model using a pair-based or hybrid approach.%
\footnote{Indeed, the importance of partnership concurrency in HIV transmission
  has been discussed extensively \cite{Boily2012}.}
Thus, EPA represents an alternative to hybrid / pair-based models for such contexts,
and a new solution to a longstanding modelling challenge \cite{Dietz1988sti}.
\par % future work --------------------------------------------------------------------------------
We have identified four key areas of future work.
First, we developed EPA as part of the Eswatini HIV transmission model,
which includes 8 risk groups and 4 partnership types;
a simpler model --- \eg with 2 risk groups and 2 partnerships types ---
may allow more precise understanding of model dynamics under different conditions,
and could be used as a reference implementation for other modellers.
To this end, we have developed such a model with code online.%
\footnote{\hreftt{github.com/mishra-lab/epa-model-toy}}
Second, it would be helpful to explore in more detail which model parameters
can compensate for differences between force of infection approaches
during model calibration to common targets,
which may be better studied using the simpler model.
Third, EPA should be compared and validated against
``gold standard'' individual-based models, as well as pair-based and hybrid models,
similar to experiments in \cite{Johnson2016mf}.
Finally, while we have focused on HIV here,
approximation of sexual partnership dynamics is also relevant for
modelling other sexually transmitted infections \cite{??}.
Indeed, approximation of repeated contact dynamics in general is likely relevant for
modelling a broad range of infectious diseases \cite{Pung2024temp}.
However, careful consideration should be given to
infections with recovery and/or re-infection,
as we have not yet considered how such processes
should be modelled within the EPA framework.
