\section{Purpose}
For decades, standard compartmental models of sexually transmitted infections
have simulated sexual partnerships as ``instantaneous''.
However, this instantaneous approach may bias model dynamics,
especially those related to partnership duration.
We developed a new approach to simulating sexual partnerships in compartmental models
which moves beyond the instantaneous paradigm, and avoids such biases.
\section{Methods}
In the instantaneous approach, people change partners at least once per year,
or more frequently for shorter and/or simultaneous partnerships per-person.
Then, a fraction of people changing partners become infected,
reflecting the cumulative probability of transmission per-partnership/year (Figure~a,~left).
In the proposed approach, we define a rate of transmission per-partnership, reflecting sex frequency;
such rates can be summed across simultaneous partnerships per-person.
Then, we track the number of people who recently transmitted or acquired infection;
we decrease by one the effective numbers of partners among these people
until they form a new partnership, determined by partnership duration (Figure~a,~right).
We integrated the proposed and instantaneous approaches, separately,
into an existing model of heterosexual HIV transmission in Eswatini.
After calibrating the model to observed data under the proposed approach,
we compared modelled HIV incidence under the proposed versus the instantaneous approach,
with the same model parameters.
\section{Results}
Incidence under the instantaneous approach was
consistently greater than under the proposed approach,
with relative differences growing over time (Figure~b).
These differences can be explained mechanistically as follows.
As prevalence increases, both approaches capture population-level herd effects:
reduced onward transmission from each infection due to
\emph{new} partnerships forming between two infected people.
However, the proposed approach also captures partnership-level herd effects:
reduced transmission due to \emph{existing} partnerships
continuing between two infected people following transmission.
In the instantaneous approach, these continuing partnerships cannot be modelled,
and so infected individuals are modelled to be immediately at risk of onward transmission.
Thus, as partnership-level herd effects accumulate over time,
relative differences in incidence under the instantaneous versus the proposed approach
also grow over time.
\section{Conclusions}
Modelling sexual partnerships as instantaneous can cause
compartmental models of HIV transmission to overestimate HIV incidence,
especially in mature and declining epidemics.
The proposed approach offers a generalizable solution to move beyond instantaneous partnerships
in compartmental models of sexually transmitted infections,
and captures key epidemic dynamics related to partnership-level herd effects.
