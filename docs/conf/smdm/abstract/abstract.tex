\section{Purpose} %SM: great abstract! tough to convey easily and edits suggested largely to make it easier for reader (who may not be super super familiar with STI modeling). minor point == lots of "this" used and its a bit hard to follow what "this" refers to at times :) 
Classic compartmental models of sexually transmitted infections
simulate sexual partnerships as ``instantaneous'' by
applying the cumulative risk of transmission per-partnership
to the fraction of people forming new partnerships per unit time. 
Thus, the usual partnership model does not account for partnership duration %SM: usual or traditioanl and same term throughout
and concurrancy, and could lead to biased estimates of the contribution of
long-term partnerships to overall transmission.
We developed a new partnership model for classic compartmental models to address this methodological gap, without 
having to use an individual-based or pair-based model.
\section{Methods}
In the proposed model, we parameterize different partnerships types using:
number of concurrent partnerships per-person, frequency of sex per-partnership, partnership duration;
versus the ``instantaneous'' approach: rate of partnership change, and total sex per-partnership. %SM: do we mean total sex acts?
For each type of partnership, we introduce a new strata to track individuals %SM: I know strictly its proporotions, but might be eaiser re: language/clarity to say people?
who recently transmitted or acquired infection via the partnership type and 
have not yet formed a new partnership of the same type.
Then, in the incidence equation for each partnersthip type, we reduce by one
the effective numbers of partnerships of among those who recently acquired/transmitted. %SM: define "those people"
We compared the influence of the``instantaneous' partnership model vs the proposed model by using each 
approach separately in an existing, data-driven model of heterosexual HIV transmission in Eswatini. 
We fit the Eswatini model using the proposed approach, 
and used the fitted paramters to simulate the HIV epidemic (overall HIV incidence) with the ``instantaneous''  approach.

\section{Results}  %SM: past tense or present? 
At the start of the epidemic, HIV incidence increased earlier and faster with the proposed model, but
then slowed and peaked at a lower magnitude, as compared with the ``instantaneous'  model (Figure). 
% this finding can be explained as follows.
The difference can be explained mechanistically as follows.
The transmission potential is lower in the instantaneous model
because the \emph{probability} of transmission per-partnership-year in the instantaneous model
is less than the \emph{rate} of transmission in the proposed model,
due to survival effects (HIV transmission can only occur once per-partnership). %SM: am confused by the sentence. "due to survival effects" applies to which model? suggest rephrase for clarity
Later, as the proportion of people who already transmitted to their current partner grows
(a dynamic only captured in the proposed model),
the potential for onward transmission from each new infection in the proposed model declines,
while this potential remains unchanged in the instantaneous model.  
Thus, incidence in the proposed model becomes lower, and remains lower,
than in the instantaneous model.
\section{Conclusions}
The proposed model moves beyond ``instantaneous'' partnerships
within compartmental models of sexually transmitted infections,
and captures key epidemic dynamics related to partnership duration and concurrency.
Future work should explore differences in inferred epidemic drivers under
the proposed versus other models of sexual partnerships.
% JK: I mention "concurrency" in the intro & conclusion,
%     as the model really does capture these dynamics,
%     but clearly don't really get into those details due to the space,
%     plus, I know there are stigma issues potentially related to that word,
%     so maybe just better to leave it out?