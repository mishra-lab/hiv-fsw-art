% JK: @SM After wrapping up your edits, I asked KB to review,
% and she gave some great high-level comments (in person) around:
% - intro: emphasize how long the instananeous approach has been used for ("decades")
% - into: mention how instananeous can bias epidemic dynamics, not partnership type stuff
%     since this is what we explore in the abstract results
% - methods: describe how instananeous approach works within the methods (was in intro)
% - methods: introduce the "probability" vs "rate" concept here
%     to tie-in with the mechanism in results
% - methods: even more lay language to describe how each approach works
% - methods: "simultaneous" partnerships more accessible than "concurrent"
% - describe partnership stuff as "approaches" to avoid confusion with the transmission "model" itself
% I also completely omit certain ideas to try to keep the methods / message clear:
% - model fitting: just say "calibrated" now
% - same parameters: this was also kinda confusing since the parameters aren't exactly the same,
%     as each approach requires different parameters, so we use the transformation given in the paper
% - partnership types: I think the methods and mechanism can be explained without this idea
% I'm thinking we can always add back these details in the oral / poster
\section{Purpose}
For decades, standard compartmental models of sexually transmitted infections
have simulated sexual partnerships as ``instantaneous''.
However, this instananeous approach can bias model-based epidemic projections and analyses of interventions.
We developed a new approach to simulating sexual partnerships in compartmental models
which moves beyond the instananeous paradigm.
\section{Methods}
In the instananeous approach, a fraction of people change partners per unit time,
reflecting average partnership duration, and possibly simultaneous partnerships per-person.
Then, a fraction of people changing partners become infected,
reflecting the cumulative probability of transmission per-partnership.
In the proposed approach, we define a rate of transmission per-partnership, reflecting sex frequency;
such rates can be summed across multiple simultaneous partnerships.
Then, we track the number of people who recently transmitted or acquired infection;
we decrease by one the effective numbers of partnerships among these people,
until they form a new partnership, determined by partnership duration.
We integrated the proposed approach, and the instananeous approach,
into a calibrated model of heterosexual HIV transmission in Eswatini,
and compared modelled HIV incidence under the proposed versus the instananeous approach.
\section{Results}
% SM: past tense or present?
% JK: The dual tenses was originally intended, and I still kind of prefer it -- what do you think?
%     - past tense to describe what we observed in the incidence plot,
%     - but present tense to describe the mechanistic descriptions,
%       as these are generalizable insights that are "true then and now and forever"
%     Something about writing the second part in past tense sounds kinda wrong to me
Incidence under the proposed approach increased faster initially,
but was later surpassed by incidence under the instantaneous approach (Figure).
This difference can be explained mechanistically as follows.
Initially, transmission is faster under the proposed approach
because the \emph{rate} of transmission per-partnership-year in the proposed approach
is greater than the \emph{probability} of transmission per-partnership-year in the instantaneous approach,
due to survival effects in the latter (HIV transmission can only occur once per-partnership).
% JK: still struggling for any simpler way to say this ^
As prevalence increases, both approaches capture population-level herd effects:
reduced onward transmission from each infection due to
\emph{new} parnterships forming between two infected people (assumed to be at random).
However, the proposed approach also captures partnership-level herd effects:
reduced transmission due to \emph{existing} partnerships
continuing between two infected people following transmission.
Thus, the proposed approach yields lower incidence versus the instananeous approach,
after initial epidemic growth.
% JK: previous comments on ^ below:
% SM: not entirely true as currently written, so tried to edit a bit.
%     i.e. still have herd effects with instantansouse model.
%     so the latter part of the phrase might be misleading b/c we do still see
%     the peak and fall due to natural dynamics too in an instantaneous partnership model.
% JK: excellent point! I have revised substantially to introduce / distinguish between
%     "population-level" herd effects and "partnership-level" herd effects to clarify this point
%     (and even included the latter in the title) --- what do you think?
\section{Conclusions}
Modelling sexual partnerships as instananeous can cause
compartmental models of HIV transmission to
underestimate early epidemic growth, and
overestimate HIV incidence in mature and declining epidemics.
% JK: Ooh, much better summary!
The proposed approach offers a generalizable solution to move beyond instantaneous partnerships
in compartmental models of sexually transmitted infections,
and captures key epidemic dynamics related to partnership-level herd effects.
% JK: was: "without having to use an individual-based or pair-based model."
%     I might avoid introducing this idea here (for the first time in the abstract)
%     as I don't think most readers would know about these different types of modles
%     and/or how they also solve the instananeous partnership problem.
% SM: figure - I don't understand the 2nd panel re: what "differnce vs. proposed" means.
%     Could quite follow what the 2nd panel was trying to tell me...?
% JK: Sorry, it's just the difference in incidence between the two models
%     I've revised the labels a bit & updated the colours to hopefully clarify.