\section{Purpose}
Classic compartmental models of sexually transmitted infections
model sexual partnerships as ``instantaneous'',
applying the cumulative risk of transmission per-partnership
to the fraction of people forming new partnerships per unit time.
This approach can bias the model-estimated contribution of
long-term partnerships and partnership concurrency to overall transmission.
We sought to develop a new model of sexual partnerships in classic compartmental models,
which better reflects dynamics related to partnership duration and concurrency.
\section{Methods}
In the proposed model, we parameterize partnerships using:
the number of concurrent partnerships per-person, frequency of sex per-partnership, and partnership duration;
rather than the usual: rate of partnership change, and total sex per-partnership.
We introduce a new population stratification to track the proportions of people
who recently transmitted or acquired infection via each partnership type,
and who have not yet formed a new partnership of that type.
Then, in the incidence equation, the effective numbers of
partnerships of that type among those people are reduced by one.
We integrated the proposed model, and a popular instantaneous partnership model,
into an existing model of heterosexual HIV transmission in Eswatini,
and compared overall HIV incidence in each model with the same parameters.
Parameters were drawn from the top 1000 of 100,000 model fits by likelihood under the proposed model.
\section{Results}
Incidence in the proposed model increased faster initially, but then slowed and was surpassed by
incidence in the instantaneous model of sexual partnerships (Figure).
% this finding can be explained as follows.
Initially, transmission is lower in the instantaneous model,
because the \emph{probability} of transmission per-partnership-year in the instantaneous model
is less than the \emph{rate} of transmission in the proposed model,
due to survival effects (HIV transmission can only occur once per-partnership).
Later, as the proportion of people who already transmitted to their current partner grows
(a dynamic only captured in the proposed model),
the potential for onward transmission from each new infection in the proposed model declines,
while this potential remains unchanged in the instantaneous model.
Thus, incidence in the proposed model becomes lower, and remains lower,
than in the instantaneous model.
\section{Conclusions}
The proposed model moves beyond ``instantaneous'' partnerships
within compartmental models of sexually transmitted infections,
and captures key epidemic dynamics related to partnership duration and concurrency.
Future work should explore differences in inferred epidemic drivers under
the proposed and other models of sexual partnerships.
