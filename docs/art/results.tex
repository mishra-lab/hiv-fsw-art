\section{Results}\label{art.res}
We first summarize modelled patterns of HIV transmission in the base case,
calibrated to reflect the Eswatini epidemic up to 2021.
Our model suggests that transmission within repeat sex work partnerships
was a dominant driver of early epidemic growth
(Figures \ref{fig:wiw.base.ptr}~and~\ref{fig:wiw.base.alluvial}).
However, from approximately 1994 onward,
the majority of new yearly infections were transmitted within casual partnerships, % MAN
including 50\% (median) of infections in 2020 in the base case.% % MAN
\footnote{In our model, casual partnerships can be formed by any subpopulation,
  and these partnerships subsume transactional partnerships.}
% By 2020, the largest proportion of cumulative infections acquired
% were among women with lowest sexual activity (Figure~\ref{fig:wiw.base.to}),
% while the largest proportion of cumulative infections transmitted
% were from clients (Figure~\ref{fig:wiw.base.from}).
% JK: propose to remove the above because:
%     - potentially stigmatizing
%     - a bit confusing without noting it is likely due to population sizes
%     - doesn't contribute to the story
%     - word count
Overall HIV prevalence in 2020 was median (95\% CI):
23.8~(22.4,~24.7)\,\% (Figure~\ref{fig:fit.prev}), % MAN
and overall incidence was 6.6~(5.3,~7.6) per 1000 person-years (Figure~\ref{fig:fit.inc}).
The prevalence ratio between FSW and women overall was 1.78~(1.70,~1.87), % MAN
and between clients and men overall it was 1.92~(1.49,~2.49) % MAN
(Figure~\ref{fig:fit.prev1v2}).
Due to turnover and higher HIV incidence among FSW,
achieving similar rates of diagnosis among FSW \vs other women (Figure~\ref{fig:fit.dx.rate})
required approximately twice the rate of testing. % MAN
Sex work contributed a growing proportion of infections
over 2010--2020: from 17\% to 32\% (Figure~\ref{fig:wiw.base.ptr}). % MAN
%===================================================================================================
\subsection{Objective~1: Influence of cascade differences between subpopulations}\label{art.res.1}
Figure~\ref{fig:art.1.cascade} illustrates ART cascade attainment over time
in the base case (\cashi overall by 2020) and
each of the four counterfactual scenarios (\casmd overall by 2020), while
Figure~\ref{fig:art.1.inc} illustrates overall HIV incidence in each scenario.
Figure~\ref{fig:art.1.rai} then illustrates
cumulative additional infections (CAI) and additional incidence rate (AIR)
in each counterfactual scenario \vs the base case.
% JK: rephrased results below for objectives reframing
If ART scale-up in Eswatini had been slower but relatively equal,
we estimate there would have been
8.8~(6.3,~10.9)\,\% CAI \vs the base case by 2020. % MAN
By contrast, if ART scale-up up had been slower and
disproportionately left behind FSW and clients,
we estimate there would have been
14.3~(10.8,~18.6)\,\% CAI \vs the base case by 2020 % MAN
--- a 63~(31,~128)\,\% increase. % MAN
Leaving behind either FSW or clients resulted in similar
10.9~(8.4,~13.3)\,\% or 12.4~(9.8,~14.6)\,\% CAI
\vs the base case, respectively. % MAN
Results were similar for AIR.
In all counterfactual scenarios, the majority of additional infections were
transmitted via casual partnerships (Figure~\ref{fig:art.1.wiw.ptr}) % MAN
and acquired among non-FSW women (Figure~\ref{fig:art.1.wiw.to}). % MAN
Patterns of onward transmission were also similar across scenarios % MAN
(Figure~\ref{fig:art.1.wiw.from}),
though subpopulation contributions increased if they were left behind.
\begin{figure}[h]
  \centering\includegraphics[width=.7\linewidth]{art.1.rai}
  \caption{Relative additional infections under counterfactual scenarios \vs the base case}
  \label{fig:art.1.rai}
  \floatfoot{\ffart; \ffbox.}
\end{figure}
%===================================================================================================
\subsection{Objective~2: Conditions that maximize the influence of cascade differences}\label{art.res.2}
The fitted regression models \eqref{eq:art.2.glm} indicated that
population-overall viral non-suppression ($D$) and
relative non-suppression among FSW and clients ($d_i$)
each had strong and positive effects on 2020 CAI and AIR outcomes ($p < 10^{-5}$), % MAN
corroborating the results of Objective~\ref{obj:art.1}.
Figure~\ref{fig:art.2} plots the estimated effects of
subpopulation-specific non-suppression $d_i$, plus
effect modification by epidemic conditions $C_j$.
% MAN below
The effect of non-suppression among FSW increased
with FSW population size for both CAI and AIR, and
with client population size and HIV incidence ratio \vs men overall for AIR.
The effect of non-suppression among clients increased
with client population size and incidence ratio for both CAI and AIR.
Durations buying or selling sex did not appear to modify
the impact of non-suppression among either FSW or clients,
and neither did HIV incidence ratio among FSW \vs women overall.
\begin{figure}[h]
  \subcapoverlap\centering
  \vskip1ex
  \begin{subfigure}{.7\linewidth}
    \includegraphics[width=\linewidth]{art.2.cai}
    \caption{\raggedright}
    \label{fig:art.2.cai}
  \end{subfigure}
  \vskip1ex
  \begin{subfigure}{.7\linewidth}
    \includegraphics[width=\linewidth]{art.2.air}
    \caption{\raggedright}
    \label{fig:art.2.air}
  \end{subfigure}
  \caption{Estimated effects on relative additional infections
    of disproportionate viral non-suppression ($d$) among FSW and clients \vs population overall,
    plus effect modification by epidemic conditions}
  \label{fig:art.2}
  \floatfoot{
    \sfref{fig:art.2.cai} cumulative additional infections,
    \sfref{fig:art.2.air} additional incidence rate
    by 2030 \vs base case;
    FSW: female sex workers; Clients: of FSW;
    IR: incidence ratio in 2020;
    $d_i$: difference in subpopulation-$i$-specific viral non-suppression
      \vs population overall within counterfactual scenario;
  \ffpbar[effect estimated via \eqref{eq:art.2.glm}].}
\end{figure}
