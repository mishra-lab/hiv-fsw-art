\section{Results}\label{res}
% TODO: big result update
\newcommand{\xci}[1]{xx~(xx,~xx)\xspace}
Early epidemic emergence was driven by regular sex work partnerships
(Figures \ref{fig:wiw.base.part}~and~\ref{fig:wiw.base.alluvial}).
However, casual partnerships contributed the majority of infections by 1994, % MAN (1997)
including 60\% (median) of new infections in 2020 in the base case. % MAN
By 2020, clients of FSW had transmitted the most infections (Figure~\ref{fig:wiw.base.from})
and lower risk women had acquired the most infections (Figure~\ref{fig:wiw.base.to}).
Overall HIV prevalence in 2020 was median (95\% confidence interval):
\xci{bc/prev.all.2020}\% (Figure~\ref{fig:fit.prev}),
and overall incidence was \xci{bc/inc.all.2020} per 1000 person-years (Figure~\ref{fig:fit.inc}).
The prevalence ratio among FSW versus lower risk women was \xci{bc/pr.fsw.2020},
and among clients of FSW versus lower risk men it was \xci{bc/pr.cli.2020} (Figure~\ref{fig:fit.prev1v2}).
Due to turnover and higher HIV incidence among FSW,
achieving similar rates of diagnosis among FSW versus other women (Figure~\ref{fig:fit.dx.rate})
required \xci{Rdx.fsw} times the rate of testing.
Sex work contributed a growing proportion of infections
over 2020--2040: from 23\% to 28\% (Figure~\ref{fig:wiw.base.part}). % MAN
%===================================================================================================
\subsection{Objective~1: Influence of cascade differences between risk groups}\label{res.obj.1}
Figure~\ref{fig:obj.1.cascade} illustrates cascade attainment over time
in each of the four counterfactual scenarios (\casmd overall by 2020),
plus the base case (\cashi overall by 2020).
Figure~\ref{fig:obj.1.inc} illustrates overall HIV incidence in each scenario.
Figure~\ref{fig:obj.1.rai} then illustrates
cumulative additional infections (CAI) and additional incidence rate (AIR)
in each counterfactual scenario \vs the base case.
Leaving behind both FSW and clients resulted the most additional infections: median [IQR]
28.8~[17.5,~46.2]\,\% more than the base case by 2030. % MAN
By contrast, leaving behind neither FSW nor clients resulted in the fewest additional infections:
13.0~[6.1,~25.6]\,\% more than the base case by 2030 --- % MAN
a 54.2~[30.3,~73.2]\,\% reduction. % MAN
Leaving behind either FSW or clients resulted in a similar number of additional infections:
21.8~[12.5,~36.7]\,\% and 20.4~[11.8,~34.7]\,\%, respectively. % MAN
Relative differences were similar for additional incidence rate.
Which risk groups acquired additional infections differed across scenarios
(Figure~\ref{fig:obj.1.wiw.to}),
with more additional infections among clients when FSW were left behind,
\vs among lower risk risk women when clients were left behind.
The majority of additional infections were transmitted
via casual partnerships in all scenarios (Figure~\ref{fig:obj.1.wiw.part}). % MAN
\begin{figure}[h]
  \centering\includegraphics[width=.8\linewidth]{art.1.rai}
  \caption{Relative additional infections under counterfactual scenarios \vs the base case}
  \label{fig:obj.1.rai}
  \floatfoot{\ffart; \ffbox.}
\end{figure}
%===================================================================================================
\subsection{Objective~2: Conditions that maximize the influence of cascade differences}\label{res.obj.2}
The fitted regression models \eqref{eq:obj.2.glm} indicated that
population-overall viral unsuppression ($D$) and
group-specific unsuppression among FSW and clients ($d_i$)
were each strongly and positively associated with the 2030 CAI and AIR outcomes ($p < 10^{-5}$).
These associations support the results of Objective~\ref{obj:1}.
Figure~\ref{fig:obj.2} plots the estimated effects of
group-specific unsuppression $d_i$, and
effect modification by epidemic conditions $C_j$.
The effect of unsuppression among FSW on CAI and AIR increased with: % MAN
FSW and client population sizes, client turnover, and
HIV incidence ratio among FSW \vs other women.
The effect on AIR also decreased with: HIV incidence ratio among clients \vs other men.
The effect of unsuppression among clients on CAI and AIR increased with:
FSW and client population sizes and FSW turnover.
The effect on AIR also modestly with:
HIV incidence ratio among FSW \vs other women and client turnover.
\begin{figure}[h]
  \subcapoverlap\centering
  \vskip1ex
  \begin{subfigure}{.75\linewidth}
    \includegraphics[width=\linewidth]{art.2.cai}
    \caption{\raggedright}
    \label{fig:obj.2.cai}
  \end{subfigure}
  \vskip1ex
  \begin{subfigure}{.75\linewidth}
    \includegraphics[width=\linewidth]{art.2.air}
    \caption{\raggedright}
    \label{fig:obj.2.air}
  \end{subfigure}
  \caption{Estimated effects on relative additional infections
    of disproportionate viral unsuppression ($d$) among FSW and clients \vs population overall,
    plus effect modification by epidemic conditions}
  \label{fig:obj.2}
  \floatfoot{
    \sfref{fig:obj.2.cai} cumulative additional infections,
    \sfref{fig:obj.2.air} additional incidence rate
    by 2030 \vs base case;
    FSW: female sex workers; Clients: of FSW;
    IR: incidence ratio in 2000;
    $d_i$: difference in group-$i$-specific viral unsuppression
      \vs population overall within counterfactual scenario;
  \ffpbar[effect estimated via \eqref{eq:obj.2.glm}].}
\end{figure}
\pagebreak % TEMP
