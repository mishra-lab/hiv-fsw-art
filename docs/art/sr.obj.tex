\section{Objectives}\label{sr.art}
%===================================================================================================
\subsection{Objective~1}\label{sr.art.1}
Figure~\ref{fig:art.1.cascade} illustrates cascade attainment over time across
the base case and counterfactual scenarios (Table~\ref{tab:art.1.scen})
for FSW, clients, everyone else, and the population overall.
Transient declines in VLS among treated around 2010 correspond to expanding ART eligibility.
\par
Figure~\ref{fig:art.1.inc} illustrates overall HIV incidence over time across scenarios.
As in \sref{sr.cal.wiw}, Figure~\ref{fig:art.1.wiw} illustrates
the distributions of additional infections over time \vs the base case across
counterfactual scenarios  (Table~\ref{tab:art.1.scen}),
stratified by partnership type, transmitting group, and acquiring group.
\begin{figure}[b]
  \centerline{\includegraphics[scale=.8]{art.1.inc}}
  \caption{Overall HIV incidence over time across scenarios}
  \label{fig:art.1.inc}
  \floatfoot{\ffart; \ffribbon.}
\end{figure}
\begin{figure}
  \centerline{\includegraphics[scale=.8]{art.1.cascade}}
  \caption{Cascade attainment over time across scenarios}
  \label{fig:art.1.cascade}
  \floatfoot{\ffpopz; \ffcas; \ffart; \ffribbon.}
\end{figure}
\begin{figure}
  \subcapoverlap
  \foreach \var in {ptr,from,to}{
  \begin{subfigure}{\linewidth}
    \includegraphics[width=\linewidth]{art.1.wiw.\var}
    \caption{\raggedright}
    \label{fig:art.1.wiw.\var}
  \end{subfigure}}
  \caption{Additional infections in each counterfactual scenario
    \vs the base case, stratified by:
    \sfref{fig:art.1.wiw.ptr} partnership type,
    \sfref{fig:art.1.wiw.from} transmitting group, and
    \sfref{fig:art.1.wiw.to} acquiring group}
  \label{fig:art.1.wiw}
  \floatfoot{\ffart; \ffwiw.}
\end{figure}
%===================================================================================================
\subsection{Objective~2}\label{sr.art.2}
Figure~\ref{fig:art.2.cascade} illustrates the distributions of cascade attainment by 2020
for FSW, clients, everyone else, and the population overall,
in the 10,000 randomly generated counterfactual scenarios
for the Objective~\ref{obj:art.2} regression analysis.
Figure~\ref{fig:art.2.dag} illustrates
the corresponding directed acyclic graph as reflected in \eqref{eq:art.2.glm}.
Figure~\ref{fig:art.2.r} then illustrates the simulated \vs regression-estimated
cumulative additional infections (CAI) and additional incidence rate (AIR),
which supports the use of linear regression
despite minor heteroskedasticity.
\begin{figure}[h]
  \centering
  \includegraphics[scale=.8]{art.2.cascade}
  \caption{Cascade attainment by 2020 across 10,000 randomly generated counterfactual scenarios}
  \label{fig:art.2.cascade}
  \floatfoot{\ffpopz; \ffcas; \ffbox.}
\end{figure}
\begin{figure}
  \centering\includegraphics[scale=1]{art.2.dag}
  \caption{Directed acyclic graph (DAG) for inferring
    the epidemic conditions under which
    differential viral suppression across subpopulations matters most}
  \label{fig:art.2.dag}
  \floatfoot{
    $Y$: cumulative additional infections (CAI) or additional incidence rate (AIR) by 2030;
    $D$: difference in population-overall viral non-suppression
      in counterfactual \vs base case scenario;
    $d_i$: difference in subpopulation-$i$-specific viral non-suppression
      \vs population overall within counterfactual scenario;
    $C_j$: epidemic conditions (effect modifiers of $d_i$).}
\end{figure}
\begin{figure}[h]
  \begin{subfigure}{.5\linewidth}
    \includegraphics[width=\linewidth]{art.2.cai.r}
    \caption{Cumulative additional infections}
    \label{fig:art.2.cai.r}
  \end{subfigure}%
  \begin{subfigure}{.5\linewidth}
    \includegraphics[width=\linewidth]{art.2.air.r}
    \caption{Additional incidence rate}
    \label{fig:art.2.air.r}
  \end{subfigure}
  \caption{Simulated \vs regression-estimated outcomes,
    and corresponding residuals for Objective~\ref{obj:art.2}.}
  \label{fig:art.2.r}
  \floatfoot{Outcomes computed for 2030 \vs base case;
    regression models defined in \eqref{eq:art.2.glm}.}
\end{figure}
