Eswatini achieved the UNAIDS 95-95-95 antiretroviral therapy (ART) cascade targets by 2020
--- having
95\% diagnosed among people living with HIV,
95\% treated among diagnosed, and
95\% virally suppressed among treated
---
with minimal inequalities across subpopulations,
including for female sex workers (FSW) and their clients.
We sought to estimate the additional HIV infections expected
if cascade scale-up had not been equal in Eswatini,
and under which epidemic conditions these inequalities could have the largest influence.
We first built and calibrated
a large compartmental model of heterosexual HIV transmission in Eswatini.
We then defined counterfactual scenarios in which
the population overall reached \casmd by 2020,
but where FSW, clients, both, or neither
were disproportionately left behind, reaching only \caslo.
Compared with observed cascade scale-up in Eswatini,
leaving behind neither FSW nor their clients led to median (95\% CI)
14.9 (10.4,~18.4)\% additional infections by 2030 \vs % MAN
26.3 (19.7,~33.0)\% if both were left behind % MAN
--- a 73~(40,~149)\% increase. % MAN
The influence of cascade inequalities among FSW and/or clients
was largely mediated by population sizes and relative HIV incidence.
As Eswatini has shown, addressing inequalities in the ART cascade,
particularly those that intersect with high transmission risk,
can maximize incidence reductions from cascade scale-up.
