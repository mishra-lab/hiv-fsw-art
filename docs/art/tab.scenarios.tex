\begin{tabular}{lCCCccc}
  \toprule
  & \multicolumn{3}{c}{ART cascade in 2020\tn{a}} 
  & \multicolumn{3}{c}{Re-scaled cascade rates\tn{b}} \\
  \cmidrule(rl){2-4}\cmidrule(rl){5-7}
  Scenario                                   &   FSW    & Clients  & Overall  & FSW & Clients & LR  \\
  \midrule
  \emph{Base Case}                           & \cashigh &    ---   & \cashigh & --- &   ---   & --- \\
  \emph{Leave Behind: FSW}                   & \caslow  &    ---   & \casmid  & \By &   \Bn   & \By \\
  \emph{Leave Behind: Clients}               &    ---   & \caslow  & \casmid  & \Bn &   \By   & \By \\
  \emph{Leave Behind: FSW \& Clients}        & \caslow  & \caslow  & \casmid  & \By &   \By   & \By \\
  \emph{Leave Behind: Neither}               &    ---   &    ---   & \casmid  & \Bn &   \Bn   & \By \\
  \bottomrule
\end{tabular}\floatfoot
\tnt[a]{Cascade: \% diagnosed among HIV+; \% on ART among diagnosed; \% virally suppressed among on ART};
\tnt[b]{Rates of: diagnosis; ART initiation; treatment failure}.
Notation ---
FSW: female sex workers;
Clients: of FSW;
LR: lower risk.
Figure~\ref{fig:obj.1.cascade} plots the resulting cascades over time.
% TODO: explain counterfactual scenario re-fitting details?