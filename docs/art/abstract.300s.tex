\paragraph{Background}
Inequalities in the antiretroviral therapy (ART) cascade across subpopulations
remain an ongoing challenge in the global HIV response.
Eswatini achieved the UNAIDS 95-95-95 targets by 2020,
with differentiated programs to minimize inequalities across subpopulations,
including for female sex workers (FSW) and their clients.
We sought to estimate additional HIV infections expected in Eswatini
if cascade scale-up had not been equal,
and under which epidemic conditions these inequalities could have the largest influence.
\paragraph{Methods}
Drawing on population-level and FSW-specific surveys in Eswatini,
we developed a compartmental model of heterosexual HIV transmission
which included eight subpopulations and four sexual partnership types.
We calibrated the model to stratified HIV prevalence, incidence, and ART cascade data.
Taking observed cascade scale-up in Eswatini as the base-case
--- reaching \cashi in the overall population by 2020 ---
we defined four counterfactual scenarios in which
the population overall reached \casmd by 2020,
but where FSW, clients, both, or neither
were disproportionately left behind, reaching only \caslo.
We quantified relative additional cumulative HIV infections by 2030
in counterfactual \vs base-case scenarios.
We further estimated linear effects of
viral suppression gap among FSW and clients on additional infections by 2030, plus
effect modification by FSW/client population sizes, rates of turnover, and HIV prevalence ratios.
\paragraph{Results}
Compared with the base-case scenario, leaving behind neither FSW and nor their clients
led to the fewest additional infections by 2030: median (95\% credible interval)
13.3 (9.2,~18.7)\,\% \vs 26.7 (19.7,~33.8)\,\% if both were left behind
--- a 104~(50,~167)\,\% increase.
The effect of lower cascade on additional infections was larger for clients \vs FSW, and
client population size, turnover, and prevalence ratios were the largest modifiers of both effects.
\paragraph{Conclusions}
Inequalities in the ART cascade across subpopulations
can undermine the anticipated prevention impacts of cascade scale-up.
As Eswatini has shown,
addressing inequalities in the ART cascade,
particularly those that intersect with high transmission risk,
could maximize incidence reductions from cascade scale-up.
