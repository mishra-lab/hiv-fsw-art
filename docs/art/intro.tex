\section{Introduction}\label{art.intro}
% SM: great introduction section structure! main suggestions:
%     introduce term inequality; introduce the idea of overall vs inequality in 95-95-95
% JK: thanks! I've incorporated inequality more, but it is still a bit awkward
%     comparing the focus of abstract on cascade vs 1st intro para on ART.
%     Not sure how else to structure though.
Early HIV treatment via antiretroviral therapy (ART)
is a lifesaving intervention increasing both the quantity and quality of life
\cite{Gabillard2013,Maartens2014,Danel2015,Lundgren2015init}.
A secondary benefit to early ART given Undetectable = Untransmittable (U=U) is that
% HM: Might need to spell out.
% JK: done
onward transmission risks are mitigated in serodifferent partnerships
\cite{Anglemyer2013,Cohen2016,Rodger2019}.
As such, immediate initiation of ART after HIV diagnosis
has been recommended by WHO since 2016 \cite{WHO2016art}.
Alongside expanding ART eligibility, interest has grown in
the potential population-level prevention impacts of ART --- \ie HIV incidence reductions.
This interest has motivated
numerous modelling studies \cite{Granich2009,Eaton2012,Eaton2014art,Knight2022sr} and
several large community-based trials \cite{Makhema2019,Havlir2019,Hayes2019,Iwuji2018}
of ART cascade scale-up for ``treatment as prevention'',
% JK: do we need to spell out the meaning of 'cascade' -- i.e. steps?
especially across Sub-Saharan Africa.
In general, the incidence reductions realized in these trials
% RK: realized in/by ??
% JK: done
have not met expectations from modelling, prompting questions about
the potential influence of modelling assumptions on predictions~\cite{Baral2019}.
% SB: Important to really lay out the clear differences and disconnect between
%     individual observed benefits and population-level ones.  Ie, U+U is a fact.
%     So it is really that it didn't lower HIV incidence in the trials that was something that necessitated interrogation.
% LW: Would be more explicit about the different findings from those trials vs. previous modeling studies.
% JK: Have changed 'ART prevention impacts' -> 'HIV incidence reductions'.
%     plus the revised opening to this paragraph Re. U=U.
\par
Within these modelling studies, ART prevention impacts are usually quantified as
HIV incidence rate reduction or cumulative infections averted in
scenarios with higher ART cascade attainment \vs
scenarios with lower attainment \cite{Knight2022sr}.
Modelled populations are often stratified by risk,
% HM: Risk of disease acquisition?
% JK: acquisition and/or transmission -- I think it is sufficiently implied?
including key populations like female sex workers (FSW) and their clients,
% HM: This is the first time FSW appears
% JK: added spelled out
to capture important epidemic dynamics related to risk heterogeneity
\cite{Stigum1994,Garnett1996,Watts2010}.
However, these studies generally assume that cascade
attainment (\ie proportions of people living with HIV who are diagnosed, treated, and virally suppressed) or
progression (\ie rates of diagnosis, treatment initiation, and viral suppression)
are equal across modelled subpopulations.
For example, among the modelling studies in a recent review \cite{Knight2022sr},
key populations were usually assumed to have average cascade progression,
or above average progression in some scenarios, but never below average,
as compared to the population overall.
% LW: Is this applicable to all 3 cascades? Rate of dx, tx initiation and tx discontinuation?
%     But never 'below average' made it sound like a limitation;
%     even though it may not be in the context of rate of discontinuation.
%     I would rephrase this sentence as sth below to be more specific:
%    "key populations were usually assumed to have "average" cascade progression as compared to the population overall;
%     a few studies had considered greater rates of diagnosis in xx,  greater treatment initiation in xx,
%     and/or greater treatment discontinuation in xx."
%     Cite literature in each case; so it is more explicit. For example. For treatment discontinuation,
%     I would not blame if nobody had assumed below average discontinuation rate for key population.
% JK: The discontinuation part is tricky since it does invert the association with VS attainment as you say,
%     however I think it could be confusing / distracting  to dive into these details here
%     (n.b. we don't mention 'discontinuation' yet here) vs focusing on 'viral suppression' in general
%     Some models also don't consider any 'off-ART' state, so % VS really is defined by the one rate.
\par
Yet, there is growing evidence of inequalities in ART cascade attainment across populations strata,
% SM: maybe we introduce the term inequalities here?
% LW: Echo Stef's point of consistency: ART cascade vs. HIV treatment cascade vs. ART scale-up
% JK: love it!
including age, gender, mobility, and HIV risk \cite{Hakim2018,Green2020}.
% SM: risk of?
% JK: HIV (acquisition and/or onward transmission)
These differences can be driven by
systemic barriers to engagement in care faced by marginalized populations,
% SM: i've been learning from work in the social epidemiology field not to call the barriers unique,
%     but rather systemic (to reflect the world in whicih we live in re: systems and structures shaping barriers)...
%     i.e. they are unique b/c discrimination and "othering" is created by systems...
% JK: ah, good to know -- thanks! I've replaced 'unique' -> 'systemic' throughout (I think SB flagged too)
which intersect with individual, network, and structural determinants of HIV risk,
such as economic insecurity, mobility, stigma, discrimination, and criminalization
\cite{Wanyenze2016,Schwartz2017,Schmidt-Sane2022}.
% SM: what do we mean by drivers here? HIV risk of acquistion or onward transmission or both?
% RK: Perhaps a slight expansion would be useful, outlining some well-described barriers that FSW face in accessing care?
%     Words may be too short though I guess....
% SB: I would be clear about individual, network, and structural determinants really driving this.
% JK: Have added a list of specific factors
Moreover, cascade data may be lacking among populations experiencing the greatest barriers to care
--- \ie the lowest ART cascades likely remain unmeasured \cite{Hakim2018,Boothe2021}.
% SM: not 100% sure what we want to say here? agree, would clarify ...
%     e.g. do we mean something like: "However, current data systems for monitoring the HIV cascade
%     rarely do so by populations at differential risks of HIV transmission.
%     Thus, inequalities in the HIV cascade, especially among populations most likely to be left behind
%     by current HIV programs, may go undetected despite a country acheiving 95-95-95 overall.
% SM: would set up the idea here about how can have an overall cascade,
%     but with inequalities across populations - but perhaps that is what you were trying to do?
% HM: What does this mean? The lowest cascades among whom?
% LW: Unclear what you mean by lowest cascade?
% JK: Hopefully the added text before \ie clarifies?
These intersections of risk and cascade heterogeneity
% SM: i really like this
could potentially undercut the population-level prevention impacts of ART
anticipated from individual-level and model-based studies \cite{Baral2019}.
% SB: And individual-level studies.
% JK: done
Therefore, we sought to examine the following questions
in an illustrative modelling analysis:
\begin{enumerate}
  \item\label{obj:art.1} How are estimates of population-level ART prevention impacts
    % RK: impact or impacts? Style thing, no right answer
    % JK: I've searched to ensure 'impacts'
    influenced by differences in ART cascade across subpopulations?
    % SB: Is there a different way of calling folks other than risk groups.
    %     The answer may be no, but I know there may be some tension with this language
    %     given the common stigma associated with the term "risk" and being "high risk",
    % LW: Maybe population subgroups? And somewhere we define population subgroups as subgroups at different risk of HIV...
    % JK: Have tried to replace with subpopulations throughout --
    %     the downside is we lose the connection to 'risk heterogeneity'
    %     TODO: appendix too
  \item\label{obj:art.2} Under which epidemic conditions
    do such differences have the largest influence?
\end{enumerate}
We examined these questions using
a deterministic compartmental model of heterosexual HIV transmission in Eswatini \cite{Knight2019},
focusing on risk differences related to sex work.
Eswatini has the highest national HIV prevalence in the world \cite{UNAIDS2021},
% SM: given our positionality here, I think ok to remove "but" and "outstanding"
% JK: good points!
and recently achieved large cascade gains --- surpassing 95-95-95 ---
through multiple interventions led by the MaxART Consortium \cite{Walsh2020,SHIMS3,AIDSinfo}.
As such, we used observed ART cascade scale-up in Eswatini as a \emph{base case}
reflecting evidently attainable scale-up,
and explored \emph{counterfactual} scenarios in which scale-up was slower,
and where specific populations could have been left behind.
% SM: re-read a few times, and I wonder if we should remove quotation marks around left behind;
%     quoations may imply not-real/fake/a-term-we-made-up?
% JK: sounds good -- I was waffling between these myself while writing...
% SM: something to consider for througout our paper,
%     to use term populations instead of risk groups (except in methods).
%     the term risk group is starting to be removed from HIV discussions, and Stef made a nice suggestion about this
%     in one of our grants and it read nicely to talk about populations
%     (esp b/c at times, FSW are not at highest per-capita risk of HIV acquistion due to programs! )
% JK: as noted above, have replaced with subpopulations throughout
