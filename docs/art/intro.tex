\section{Introduction}\label{art.intro}
Early HIV treatment via antiretroviral therapy (ART) is a lifesaving intervention
increasing both the quantity and quality of life \cite{Lundgren2015init}.
A secondary benefit of early ART given Undetectable = Untransmittable (U=U) is that
transmission risks are mitigated in serodifferent partnerships \cite{Cohen2016}.
To realize these benefits, massive efforts are underway to achieve
the UNAIDS 95-95-95 ART cascade targets \cite{UNAIDS2023} --- \ie to have:
95\% diagnosed among people living with HIV,
95\% on ART among those diagnosed, and
95\% virally suppressed among those on ART.
Botswana, Eswatini, Rwanda, Tanzania, and Zimbabwe
have already surpassed 95-95-95 nationally \cite{UNAIDS2023}, and
and achieving these targets is expected to help reduce HIV incidence towards elimination.
\par
Numerous transmission modelling studies have sought to estimate
the prevention impacts of achieving 90-90-90+ across Sub-Saharan Africa
\cite{Eaton2012sys,Knight2022sr}.
Modelled populations are often stratified by risk,
including key populations like female sex workers (FSW) and their clients,
to capture important epidemic dynamics related to risk heterogeneity \cite{Watts2010}.
However, these studies have generally assumed that ART cascade
attainment (\ie proportions diagnosed, treated, and virally suppressed) or
progression (\ie rates of diagnosis, treatment initiation, and viral suppression)
were equal across modelled subpopulations.
For example, among the studies in \cite{Knight2022sr} (see also Box~\ref{ric}),
key populations were usually assumed to have
equal cascade progression with the population overall,
or greater in some scenarios, but never lesser.
\par
Yet, there are growing concerns that inequalities in the ART cascade
could undermine the population-level prevention impacts of ART
anticipated from individual-level and model-based studies
\cite{Baral2019,Green2020,Maheu-Giroux2024}.
Specifically, available data suggest that cascade attainment can be lower
among subpopulations at greater risk of HIV acquisition and/or transmission,
including key populations, younger men and women, and highly mobile populations
\cite{Hakim2018,Green2020}.
These inequalities can be driven by
systemic barriers to engagement in care faced by marginalized populations,
which intersect with individual, network, and structural determinants of HIV risk,
such as economic insecurity, mobility, stigma, discrimination, and criminalization
\cite{Lancaster2016sr,Wanyenze2016,Schwartz2017,Schmidt-Sane2022}.
Moreover, cascade data may be lacking entirely
for subpopulations experiencing the greatest barriers to care
--- \ie the lowest ART cascades likely remain unmeasured \cite{Hakim2018}.
\par
Eswatini, which has had the highest national HIV prevalence in the world,
largely minimized these cascade inequalities en route to 95-95-95,
drawing on community engagement to identify and address
subpopulation-specific barriers to care \cite{Walsh2020,SHIMS3,EswIBBS2022}.
We sought to quantity the impacts of this achievement on HIV transmission.
To do so, we developed a deterministic compartmental model
of heterosexual HIV transmission, including FSW and their clients,
and calibrated this model to reflect
the observed HIV epidemic and cascade scale-up in Eswatini.
We then compared cumulative HIV infections and HIV incidence over 2000--2020
in this \emph{base case} with \emph{counterfactual} scenarios in which
cascade scale-up was slower, and where
FSW and/or their clients were disproportionately left behind
(Objective~\obj{art.1}).
We also sought to identify epidemic conditions under which
such inequalities in cascade scale-up could have the largest impact
(Objective~\obj{art.2}).
