\address{
  Peter Hayward\\
  Editor-in-Chief\\
  The Lancet HIV
}{Dr. Sharmistha Mishra\\
  MAP Centre for Urban Health Solutions\\
  St. Michael's Hospital, Unity Health Toronto\\
  University of Toronto}
Dear Peter Hayward,
% JK: @SM, I tried to incorporate 'inequalities' framing here too,
%     but in some places it was too awkward.
%     If you want/can make it work, feel free, but I also think it is OK as-is.
\par
We are pleased to submit our manuscript entitled
\emph{Intersections of risk and intervention heterogeneity:
  a modelling study on the prevention impacts of antiretroviral therapy in Eswatini}
for consideration as an \emph{Article} in \emph{The Lancet HIV}.
\par
Achieving the UNAIDS 90-90-90 / 95-95-95 treatment cascade targets
was projected by mathematical models to reduce HIV incidence substantially.
However, large-scale community-base trials of treatment as prevention
have not consistently demonstrated the anticipated reductions.
One hypothesis for why model and trial results have differed posits that
populations at highest risk of HIV acquisition and transmission might be
disproportionately left behind in efforts to rapidly increase coverage.
\par
In this paper, we explore evidence for this hypothesis,
focusing on female sex workers (FSW) and their clients in Eswatini.
Drawing on population-level and FSW-specific surveys,
we build and calibrate a large compartmental transmission model
to reflect the observed HIV epidemic and treatment cascade scale-up in Eswatini.
Then, we estimate total additional HIV infections in counterfactual scenarios where
FSW, clients, both, or neither are disproportionately left behind.
Through sensitivity analysis, we further explore which epidemic conditions modify
the effect on additional infections of cascade inequalities among FSW and clients.
\par
We find that unequal ART cascade scale-up which leaves behind FSW or their clients
results in substantially more HIV infections \vs equitable scale-up.
We also find that the impact on additional infections of
leaving behind FSW and/or clients was largely determined by
characteristics of the client population.
\par
To our knowledge, ours is the first modelling study to explore
the potential implications of differences in cascade attainment between subpopulations
within consistent population-overall attainment.
As such, our findings offer unique insights about
the importance of equitable scale-up
for the maximizing the prevention impacts of ART.
\par
Thank you for your consideration and we look forward to hearing from you.
\medskip\par
Sincerely,
\par
Jesse Knight and Sharmistha Mishra\\
on behalf of all authors
