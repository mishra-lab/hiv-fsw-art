\address{
  Prof. Kenneth H. Mayer \& Dr. Annette Sohn\\
  Editors-in-Chief\\
  Journal of the International AIDS Society
}{Dr. Sharmistha Mishra\\
  MAP Centre for Urban Health Solutions\\
  St. Michael's Hospital, Unity Health Toronto\\
  University of Toronto}
Dear Editors,
\par
We are pleased to submit our manuscript entitled
\emph{Evaluating the impact of achieving cascade equality in Eswatini:
 a modelling study on the prevention impacts of antiretroviral therapy scale-up among female sex workers and clients} %SM: to chat about -- what if we turn this around re: acheiving what was achieved?
for consideration to publish as a \emph{Research Article} in \emph{JIAS}.
\par
As countries achieve and surpass
the UNAIDS 95-95-95 antiretroviral therapy (ART) cascade targets, 
a key question has been the extent to which inequalities in the 
cascade could undermine anticipated impact. 
There is growing evidence of inequalities in cascade attainment in many contexts,
including and especially for subpopulations
at disproportionately higher risk for HIV acquisition and transmission,
such as female sex workers (FSW) and their clients.
However, to date, there remains an evidence gap on the  
potential implications of these inequalities on
the prevention impacts of ART scale-up. 
To best answer this question, a retrospective, data-informed impact evaluation with 
an epidemic model provides an opportunity to rigorously estimate the downtream impact of cascade inequalities by 
simulating "historical what if" scenarios with and without inequalities.  
Available data suggest that Eswatini achieved not only
95-95-95 among the population overall by 2020, but also similar cascade among FSW.
Thus, the sucesses achieved in Eswatini offer an unique opportunity to quantify the potential 
impact of cascade inequalities among FSW and clients. 
 
\par
In this paper, we developed and calibrated a mathematical model of HIV transmission, 
using the best available data (population-level and FSW-specific surveys), and in partnership with community leaders and program implementers.
We used the model to simulate and thus reflect, the observed HIV epidemic and ART cascade scale-up in Eswatini.
Thus, our base case uniquely reflects the achieved cascade scale-up
with minimal inequalities across subpopulations,
alongside other observed conditions (\eg increasing condom use) - thus obviating the need to make assumptions about
the future trajectories of interventions.
We then compared  %SM: revise to past-tense 
the base case scenario to four counterfactual "what if" scenarios of 
cascade inequalities.
1) weaker cascade but equal: what if the population-overall cascade had been weaker (\casmd by 2020, as in other settings) but equal across subpopulations. 
2) weaker cascade but unequal: if FSW and/or clients had been disproportionately left behind (achieving \caslo by 2020).
3) 
4)
 
Finally, we conducted sensitivity analyses to identify the epidemic conditions which influence the magnitude of downstream impact of cascade inequalities.
\par
We estimated that a weaker cascade would have led to X-X\% additional HIV infections. 
But if a weaker cascade also left behind FSW and their clients, there would have been 
20--34\% more HIV infections in Eswatini by 2030.  
%SM: to discuss, it might be easier to follow if we re-frame as " a strong and equal cascade likely averted X-X% of HIV infections"? 
The downstream impact of 
leaving behind FSW and/or clients was largely determined by
FSW and client population sizes, and FSW and client HIV incidence relative to the wider population.
\par\pagebreak % TEMP
To our knowledge, this is the first modelling study to estimate
the potential impact of cascade equality versus inequalities across subpopulations
with consistent population-overall attainment across scenarios ---
\ie all counterfactual scenarios reach the same population-overall cascade by 2020.
The findings offer unique, data- and community-informed modelling insights,
using a real-world context where FSW cascade equality was achieved,
about the importance of equitable scale-up
for the maximizing the prevention impacts of ART.
\par
Thank you for your consideration and we look forward to hearing from you.
\medskip\par
Sincerely,
\par
Jesse Knight, PhD \& Sharmistha Mishra, MD, PhD\\
on behalf of all authors
