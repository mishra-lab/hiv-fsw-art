\address{
  Dr. Till W. B\"{a}rnighausen\\
  Senior Editor\\
  PLOS Medicine
}{Dr. Sharmistha Mishra\\
  MAP Centre for Urban Health Solutions\\
  St. Michael's Hospital, Unity Health Toronto\\
  University of Toronto}
Dear Dr. B\"{a}rnighausen,
\par
We are pleased to submit our manuscript entitled
\emph{Quantifying the impact of cascade inequalities:
  a modelling study on the prevention impacts of antiretroviral therapy scale-up in Eswatini}
for consideration to publish as an \emph{Article} in \emph{PLOS Medicine}.
\par
Five countries have now achieved
the UNAIDS 95-95-95 antiretroviral therapy (ART) cascade targets:
Botswana, Eswatini, Rwanda, Tanzania, and Zimbabwe.
Achieving these targets was projected by
mathematical models to reduce HIV incidence towards local elimination.
However, alongside efforts to rapidly increase coverage,
evidence is mounting of inequalities in cascade attainment in some contexts,
including and especially for subpopulations
at disproportionately higher risk for HIV acquisition and transmission,
such as female sex workers (FSW) and their clients.
The potential implications of these inequalities on
the prevention impacts of ART scale-up have not been examined.
\par
In this paper, we examine the potential impact of
cascade inequalities on HIV transmission in the population overall,
focusing on inequalities among FSW and clients in Eswatini.
Drawing on population-level and FSW-specific surveys,
as well as insights from local community leaders and program implementers,
we build and calibrate a large compartmental transmission model
to reflect the observed HIV epidemic and ART cascade scale-up in Eswatini.
Available data suggest that Eswatini achieved not only
95-95-95 among the population overall by 2020, but also similar cascade among FSW.
Thus, our base case uniquely reflects achievable cascade scale-up
with minimal inequalities across subpopulations,
alongside other observed conditions (\eg increasing condom use).
That is, we did not need to make assumptions about
the future trajectories of interventions.
Then, as a retrospective impact evaluation, we compare
this base case scenario to four counterfactual scenarios in which
cascade inequalities were not addressed.
Specifically, we examined what would have happened to overall HIV infections over time,
if the population-overall cascade had been weaker (\casmd by 2020, as in other settings),
and if FSW and/or clients had been disproportionately left behind (achieving \caslo by 2020).
We further conducted sensitivity analyses to identify epidemic conditions which modify
the effect of cascade inequalities among FSW and clients on additional infections.
\par
We estimate that unequal ART cascade scale-up which left behind FSW and their clients
would have resulted in 20--34\% more HIV infections by 2030 \vs observed scale-up.
That is, addressing cascade inequalities in Eswatini
through tailored programs for FSW and other subpopulations
has helped avert a substantial proportion of infections.
Notably, the impact on additional infections of
leaving behind FSW and/or clients was largely determined by
their population sizes and HIV incidence relative to the wider population.
\par\pagebreak % TEMP
To our knowledge, this is the first modelling study to explore
the potential implications of cascade inequalities across subpopulations
with consistent population-overall attainment across scenarios ---
\ie all counterfactual scenarios reach the same population-overall cascade by 2020.
The findings offer unique, data- and community-informed modelling insights,
using a real-world context where FSW cascade equality was achieved,
about the importance of equitable scale-up
for the maximizing the prevention impacts of ART.
\par
Thank you for your consideration and we look forward to hearing from you.
\medskip\par
Sincerely,
\par
Jesse Knight, PhD \& Sharmistha Mishra, MD, PhD\\
on behalf of all authors
