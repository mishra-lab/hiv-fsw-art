\paragraph{Background}
% SB: I'm not sure about this start--I think could rub people the wrong way.
%     U=U is a fact at an individual level so need to differentiate somehow
%     about the population level prevention impacts.
% LW: Agree. Not only individual-level is a fact; there should also be
%     a lot of population-level prevention impact studies too.
%     so need to be a bit more specific about what is the gap.
%     Do u mean unclear in Eswatini?
% JK: have restructured based on SM edits, focusing on inequalities
Inequalities in the antiretroviral therapy (ART) cascade across subpopulations
remain an ongoing challenge in the global HIV response.
Eswatini achieved the UNAIDS 95-95-95 targets by 2020,
with differentiated programs to minimize inequalities across subpopulations,
including for female sex workers (FSW) and their clients.
We sought to estimate additional HIV infections expected in Eswatini
if cascade scale-up had not been equal,
% SM: am worried projections might imply future and not past? how about estimated prevention impacts of ART?
% JK: agree -- but I had to rephrase since the counterfactual scenarios don't reach 95-95-95
%     and tried to tie in inequalities again -- what do you think?
and under which epidemic conditions these inequalities could have the largest influence.
\paragraph{Methods}
Drawing on population-level and FSW-specific surveys in Eswatini,
we developed a compartmental model of heterosexual HIV transmission
which included eight subpopulations and four sexual partnership types.
We calibrated the model to stratified HIV prevalence, incidence, and ART cascade data.
% SB: Would keep language consistent--above said treatment cascade (and I added HIV).
%     The key is just being super consistent so clarity.
% JK: I'll try to say (HIV) treatment cascade throughout
% LW: Did you calibrate to the overall as well? If so – can mentioned
% JK: We did, but might not mention in the abstract
%     since there were also several prevalence ratios etc. and it could get messy to explain all
%     In fact, due to word limit, I am shortening further...
Taking observed cascade scale-up in Eswatini as the base-case
% SM: nice and clear
% JK: thanks!
--- reaching \cashi in the overall population by 2020 ---
we defined four counterfactual scenarios in which
the population overall reached \casmd by 2020,
% SS: The background called out 95-95-95 - but perhaps shouldn’t as then it seems strange that this is different
% JK: the 95-95-95 is still the base case for comparison
%     but we need to have lower cascade overall in counterfactuals
%     so that subpops can be lower while keeping overall fixed
but where FSW, clients, both, or neither
were disproportionately left behind, reaching only \caslo.
% LW: I would put the observed numbers here: u mentioned it surpassed 95-95-95 in Intro)
% JK: not sure what you mean? 60-40-80 was not observed but defined for the scenario
We then quantified relative additional cumulative HIV infections by 2030
in counterfactual \vs base-case scenarios.
We further estimated linear effects of
viral suppression gap among FSW and clients on additional infections by 2030, plus
effect modification by FSW/client population sizes, rates of turnover, and HIV prevalence ratios.
\paragraph{Findings}
Compared with the base-case scenario, leaving behind FSW and their clients
% SS: Ok now I'm confused, is the base-case 95-95-95 or actual levels?
% JK: The data suggests that 95-95-95 were the actual levels
led to the most additional infections by 2030: median (95\% credible interval)
26.7 (19.7,~33.8)\% \vs 13.3 (9.2,~18.7)\% if neither were left behind
--- a 50.9 (33.5,~62.5)\% reduction.
The effect of lower cascade on additional infections was larger for clients \vs FSW, and
% LW: The same level of gap in clients had a larger impact than that in FSW? Or the observed VS suppression gap
%     e.g., was larger in clients than that in FSW, therefore have a larger impact? Worth clarifying
% JK: The effect itself was larger (due to larger population size)
client population size, turnover, and prevalence ratios were the largest modifiers of both effects.
% SM: not clear from reading sentence, what "both effects" refer to?
% LW: What do both effects refer to?
% JK: effects = effect of viral suppression gap among 1) FSW and 2) clients on additional infections
%     i.e. the 'linear effects' from methods above.
%     After rephrasing the first part, maybe it is more clear?
\paragraph{Interpretation}
Inequalities in the ART cascade across subpopulations
% SM: the term "disproprotionate gaps" is a bit unclear here - do we mean gaps between FSW and clients, or between whom?
%     I think we can get through easier to reader by using term "inequalities"
%     (like some suggestions in the background section re:
%     Inequalities in viral suppression / cascade among populations at higher risk...);
%     also why just viral suppression in conclustion statement but 95-95-95 in intro
%     (i know gets to the same thing, but then perhaps make the viral suppresion part
%     specific up front in background section)?
% JK: I am really liking the 'inequalities' framing throughout
can undermine the anticipated prevention impacts of cascade scale-up.
Identifying and addressing inequalities in the ART cascade,
particularly those that intersect with high onward transmission risk,
could maximize incidence reductions from cascade scale-up.
% SM: but Eswatini achieved so....?
% JK: not sure what you mean?
% alongside continued investment in new and established prevention tools.
% SM: why only established prevention tools?
% JK: thinking more about this -- we didn't explore prevention tools in counterfactuals,
%     and very tight on words, so maybe remove this?
\paragraph{Funding}
Natural Sciences and Engineering Research Council of Canada;
Ontario Ministry of Colleges and Universities;
Canadian Institutes of Health Research;
National Institute of Allergy and Infectious Diseases.
