\paragraph{Background}
The prevention impacts of rapid antiretroviral therapy (ART) scale-up
to reach UNAIDS 95-95-95 targets remains unclear.
This scale-up may also ``leave behind'' populations at highest risk for HIV,
including female sex workers (FSW) and their clients.
We sought to examine
how projections of ART prevention impacts are influenced by
differences in treatment cascade across risk groups, and
under which epidemic conditions such differences have the largest influence.
\paragraph{Methods}
Drawing on population-level and FSW-specific surveys in Eswatini,
we developed a 1008-compartment model of HIV transmission
which included eight risk groups and four sexual partnership types.
We calibrated the model to HIV prevalence, incidence, and cascade data,
stratified by risk group where possible.
Taking observed ART scale-up in Eswatini as the base-case
--- reaching \cashi in the population overall by 2020 ---
we defined four counterfactual scenarios in which
the population overall reached \casmd by 2020,
but where FSW, clients, both, or neither
were disproportionately ``left behind'', reaching only \caslo.
We then quantified relative additional HIV infections by 2030
in counterfactual \vs base-case scenarios.
We further estimated linear effects of
viral suppression gap among FSW and clients on additional infections by 2030, plus
effect modification by FSW/client population sizes, rates of turnover, and HIV prevalence ratios.
\paragraph{Findings}
Compared with the base-case, ``leaving behind'' FSW and their clients
yielded the most additional infections by 2030: median (95\% credible interval)
26.7 (19.7,~33.8)\% \vs 13.3 (9.2,~18.7)\% if neither were ``left behind''
--- a 50.9 (33.5,~62.5)\% reduction.
Viral suppression gap among clients \vs FSW had a larger effect on additional infections, and
client population size, turnover, and prevalence ratios were the largest modifiers of both effects.
\paragraph{Interpretation}
Disproportionate viral suppression gaps among populations at higher risk for HIV,
including FSW and their clients,
can undermine the anticipated prevention impacts of ART scale-up.
Efforts to identify and address such gaps are needed to maximize incidence reductions,
alongside continued investment in established prevention tools.
\paragraph{Funding}
% TODO: (!) NIH?
Natural Sciences and Engineering Research Council of Canada;
Canadian Institutes of Health Research;
Ontario Ministry of Colleges and Universities.
