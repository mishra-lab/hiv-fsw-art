\paragraph{Background}
% SB: I'm not sure about this start--I think could rub people the wrong way.
%     U=U is a fact at an individual level so need to differentiate somehow
%     about the population level prevention impacts.
% LW: Agree. Not only individual-level is a fact; there should also be
%     a lot of population-level prevention impact studies too.
%     so need to be a bit more specific about what is the gap.
%     Do u mean unclear in Eswatini?
Inequalities in the HIV cascade across populations remain
an ongoing challenge in the global HIV response.
Eswatini achieved the UNAIDS 95-95-95 targets by XXXX,
with programs for female sex workers (FSW) that also ensured
FSW living with HIV also reached 95-95-95.
We sought to examine what would have happened if the scale up of
antiretroviral therapy (ART) to reach 95-95-95 overall also 
``left behind'' populations at highest risk for HIV, including FSW and their male clients.
We examined how projections of ART prevention impacts could be influenced by
% SM: am worried projections might imply future and not past? how about estimated prevention impacts of ART?
differences in HIV treatment cascades across populations, and
under which epidemic conditions such differences have the largest influence.
\paragraph{Methods}
Drawing on population-level and FSW-specific surveys in Eswatini,
we developed a compartmental model of heterosexual HIV transmission
which included eight risk groups and four sexual partnership types.
We calibrated the model to HIV prevalence, incidence, and cascade data,
% SB: Would keep language consistent--above said treatment cascade (and I added HIV).
%     The key is just being super consistent so clarity.
stratified by risk group where data are available.
% LW: Did you calibrate to the overall as well? If so – can mentioned
Taking observed levels of HIV treatment cascade in Eswatini as the base-case
% SM: nice and clear
--- reaching \cashi in the overall population by 2020 ---
we defined four counterfactual scenarios in which
the population overall reached \casmd by 2020,
% SS: The background called out 95-95-95 - but perhaps shouldn’t as then it seems strange that this is different
but where FSW, clients, both, or neither
were disproportionately ``left behind'', reaching only \caslo.
% LW: I would put the observed numbers here: u mentioned it surpassed 95-95-95 in Intro)
We then quantified relative additional HIV infections by 2030
in counterfactual \vs base-case scenarios.
We further estimated linear effects of
viral suppression gap among FSW and clients on additional infections by 2030, plus
effect modification by FSW/client population sizes, rates of turnover, and HIV prevalence ratios.
\paragraph{Findings}
Compared with the base-case scenario, ``leaving behind'' FSW and their clients
% SS: Ok now I'm confused, is the base-case 95-95-95 or actual levels?
led to the most additional infections by 2030: median (95\% credible interval)
26.7 (19.7,~33.8)\% \vs 13.3 (9.2,~18.7)\% if neither were ``left behind''
--- a 50.9 (33.5,~62.5)\% reduction.
The viral suppression gap among clients \vs FSW had a larger effect on additional infections.
% LW: The same level of gap in clients had a larger impact than that in FSW? Or the observed VS suppression gap
%     e.g., was larger in clients than that in FSW, therefore have a larger impact? Worth clarifying
Client population size, turnover, and prevalence ratios were the largest modifiers of both effects.
% SM: not clear from reading sentence, what "both effects" refer to?
% LW: What do both effects refer to?
\paragraph{Interpretation}
Inequalties in the HIV cascade among populations at higher risk for HIV, including FSW and their clients,
% SM: the term "disproprotionate gaps" is a bit unclear here - do we mean gaps between FSW and clients, or between whom?
%     I think we can get through easier to reader by using term "inequalities"
%     (like some suggestions in the background section re: Inequalities in viral suppression / cascade among populations at higher risk...);
%     also why just viral suppression in conclustion statement but 95-95-95 in intro
%     (i know gets to the same thing, but then perhaps make the viral suppresion part specific up front in background section)?
can undermine the anticipated prevention impacts of ART scale-up.
Identifying and addressing inequalities in the HIV cascade,
particularly at the intersection of high onward transmission risks,
could maximize incidence reductions,
% SM: but Eswatini achieved so....?
alongside continued investment in established prevention tools.
% SM: why only established prevention tools?
\paragraph{Funding}
Natural Sciences and Engineering Research Council of Canada;
Ontario Ministry of Colleges and Universities;
Canadian Institutes of Health Research;
National Institute of Allergy and Infectious Diseases.
