\section{Methods}\label{art.meth}
% SM: I really liked how you succintly synthesized 3 years of work into 2 pages :)
%     with a focused and clear desciption of the methods and key elements (with full details in the appendix).
%     This will be a good example for future PhD students given the reality of the way
%     general medical/public health papers/journals are set up for now.
%     a few minor edits/comments for consideration but overall I think
%     really nicely written for a broad yet HIV-knowlegeble audience
% JK: Thanks!
We constructed a deterministic compartmental model of heterosexual HIV transmission, stratified by
subpopulations defined by sex and sexual activity, and
health states reflecting HIV natural history and ART cascade of care.
The model includes eight subpopulations,
including FSW at higher \vs lower risk, and likewise for clients of FSW,
and four partnership types, including regular and occasional sex work (Figure~\ref{fig:model}).
We calibrated the model to reflect the HIV epidemic and ART scale-up in Eswatini (\emph{base case}).
We then explored \emph{counterfactual} scenarios in which
ART cascade was reduced among various combinations of subpopulations,
and quantified ART prevention impacts by comparing \emph{base case} and \emph{counterfactual} scenarios.
%===================================================================================================
\subsection{Model Parameterization \& Calibration}\label{art.meth.par}
Complete details of the model structure, parameterization, and calibration are given in Appendix~\ref{mod}.
\paragraph{HIV}
HIV natural history included acute infection and stages defined by CD4-count.
We modelled relative rates of infectiousness by stage
as an approximation of viral load \cite{Wawer2005,Boily2009,Donnell2010,Bellan2015},
% RK: None of these were structured to capture impact of acute infection on transmission I think
%     but I assume you did something clever to include that?
% JK: Oh yes good point, I've added Bellan2015 which was the main reference for acute infection
%     but may really have to cut references for Lancet limit of 30
as well as rates of HIV-attributable mortality by stage \cite{Badri2006,Anglaret2012,Mangal2017}.
\paragraph{Risk heterogeneity}
We captured risk heterogeneity through subpopulation-level factors, including
% SM: see earlier comments/thoughts re: populations instead of groups...
% JK: done
subpopulation size, average duration in subpopulation,
genital ulcer disease (GUD) prevalence, and rates / types of partnership formation;
% LW: Spell out first occurence of "STI"
% JK: done
and partnership-level factors, including
assortative mixing, partnership duration, frequency of sex, and levels of condom use.
Table~\ref{tab:art.het} summarizes key parameter values and sampling distributions related to risk heterogeneity.
% HM: FSW of women overall and clients of men overall read odd.
%     Would just put FSW overall and clients overall as u explained in footnote.
%     Sex work clients per year -> # of clients in sex work? Similar: # of visits? GUD prevalence?
% JK: I might leave as-is, since the data and appendix have these pop sizes as % of wo/men
To parameterize FSW at higher \vs lower risk, we analyzed individual-level survey data
from Swati FSW in 2011 \cite{Baral2014} and 2014 \cite{EswKP2014} (Appendix~\ref{mod.par.fsw}).
% SB: Still ok to say Swazi even if that is what country was called at the time?
% JK: Oops, no just a typo...
We parameterized the remaining subpopulations using reported data from national studies in
2006 \cite{SDHS2006}, 2011 \cite{SHIMS1}, and 2016 \cite{SHIMS2}.
We modelled
increasing condom use (Figure~\ref{fig:fit.condom}),
increasing voluntary medical male circumcision (Figure~\ref{fig:fit.circum}), and
decreasing GUD prevalence over time,
% We assumed that only condom use varied (increased) over time (Figure~\ref{fig:fit.condom}).
% SM: clarify what "only" might mean here? could it be interpreted as testing/ART uptake did not vary over time?
% SB: We didn’t do anything with PrEP here, right?  May need to include in limitations that PrEP was not modeled.
% LW: Looks like. Agree to note in limitation section.
% SS: agree
% JK: have revised to make dynamic transmission modifiers more clear
%     and added PrEP in limitations
We did not model other interventions
(e.g. current pre-exposure prophylaxis scale-up \cite{EswCOP21})
nor non-heterosexual HIV transmission.
\begin{table}
  \centering
  \caption{Selected model parameters related to risk heterogeneity}
  \footnotesize
% TODO: update
\begin{tabular}{llrclc}
  \toprule
                          &                                &  \multicolumn{3}{c}{Prior}   &         \\
  \cmidrule(rl){3-5}
  Parameter               & Stratification                 & Mean & (95\% CI)  & Distrib. &  Ref. * \\
  \midrule
  Population size         & FSW of women overall           &  2.8 & (0.6,~6.5) & Beta     &         \\
  (\% of total)           & Clients of men overall         &   30 & (6.0,~70)  & ---      & See~XX  \\
                          & HR FSW of FSW overall          &   20 &    ---     & ---      & Assumed \\
                          & HR clients of clients overall  &   20 &    ---     & ---      & Assumed \\[1ex]
  Duration in group       & HR FSW                         &  3.6 & (1.9,~5.8) & Gamma    & See~XX  \\
  (mean years)            & LR FSW                         &   10 & (9.0,~11)  & Gamma    & See~XX  \\
                          & All clients                    &   10 & (6.0,~15)  & Gamma    &         \\[1ex]
  Relative infectiousness & Acute infection                &  5.3 & (1.0,~13)  & Gamma    &         \\
                          & Any GUD p12m                   &  2.9 & (1.0,~5.7) & Gamma    &         \\[1ex]
  Relative susceptibility & Receptive vaginal sex          & 1.45 & (1.0,~2.0) & Gamma    &         \\
                          & Receptive anal sex             &   10 &    ---     & ---      &         \\
                          & Any GUD p12m: women            &  5.3 & (1.5,~12)  & Gamma    &         \\
                          & Any GUD p12m: men              &  7.7 & (2.0,~18)  & Gamma    &         \\[1ex]
  Any GUD p12m            & LR FSW                         &   16 &  (7,~28)   & Beta     &         \\
  prevalence (\%)         & HR FSW                         &   47 &  (19,~89)  & ---      &         \\
                          & HR clients                     &   12 &  (7,~22)   & ---      &         \\
                          & Everybody else                 &    7 &    ---     & ---      &         \\[1ex]
  Sex acts per            & Main/spousal                   &   78 & (27,~156)  & Gamma    &         \\
  partnership-year        & Casual                         &   30 & (4.4,~82)  & ---      & See~XX  \\
                          & Occasional sex work            &   12 &    ---     & ---      & Assumed \\
                          & Regular sex work               &   31 &  (18,~48)  & Gamma    &         \\[1ex]
  Partnership anal sex    & Main/spousal \& casual         &  5.9 & (0.6,~17)  & Beta     &         \\
  (\% of acts)            & Occasional \& regular sex work &  9.7 & (0.6,~29)  & Beta     &         \\[1ex]
  Condom use in 2020      & Main/spousal                   &   42 &  (31,~54)  & Beta     & See~XX  \\
  (\% of acts protected)  & Casual                         &   69 &  (65,~74)  & Beta     & See~XX  \\
                          & Occasional sex work            &   88 &  (78,~97)  & Beta     & See~XX  \\
                          & Regular sex work               &   79 &  (64,~90)  & Beta     & See~XX  \\
                          & Anal vs vaginal sex            &   77 &  (50,~95)  & Beta     &         \\[1ex]
  Partnerships per year   & LR FSW, occasional sex work    &   49 &  (30,~72)  & Gamma    &         \\
                          & HR FSW, occasional sex work    &   98 & (58,~153)  & ---      &         \\
                          & LR FSW, regular sex work       &  101 & (73,~133)  & Gamma    &         \\
                          & HR FSW, regular sex work       &  151 & (107,~205) & ---      &         \\[1ex]
  Sex work visits         & LR clients                     &   26 &  (11,~50)  & Gamma    &         \\
  per year                & HR clients                     &   89 & (34,~174)  & Gamma    &         \\
  % TODO: mixing!
  \bottomrule
\end{tabular} 
\floatfoot{
  \ffpops;
  p12m: past 12 months.}
  \label{tab:art.het}
\end{table}
\paragraph{ART cascade}
% SS: propose replacing low/medium-risk wo/men -> "cisgender women with low/medium sexual activity" (throughout)
% JK: done
We modelled rates of HIV diagnosis among people living with HIV as monotonically increasing over time.
% SM: rates of testing, rates of testing if infected/PLHIV, or rates of actual diagnoses made
%     (i.e. rates of cases detected if there was diagnosis-based surveillance data available)?
% JK: rates of testing among PLHIV not yet diagnosed
%     i.e. rate of moving from undiagnosed -> diagnosed
We defined a base rate for women with low/medium sexual activity,
and constant relative rates for
men with low/medium sexual activity ($<1$), clients ($<1$), and FSW ($>1$),
% SS: what is this referring to? Is low clients <1 -- perhaps put the numbers for women and men separately
%     also is this <1 client per month? Per day?
% JK: sorry it was not clear, the <1 or >1 refer to the magnitude of relative rates
%     vs the base rate for low/medium activity women
reflecting increased HIV testing access via antenatal care among women \vs men,
and enhanced screening among FSW \cite{Baral2014}.
We modelled ART initiation similarly except:
the relative rate for ART initiation among FSW was $<1$,
reflecting specific barriers to uptake and engagement in care \cite{Mountain2014sr}; and
% LW: So FSW were more likely to be tested but less likely to initiate treatment?
%     Just to make sure it is in line with literature. Especially if we have data specific to Eswatini.
% JK: Good point - it might be worth allowing <1 or >1 in the next calibration,
%     since these relative rates are calibrated to ART coverage data.
%     I'd have to review again, but the types of data available for FSW vs other women
%     (e.g. % initiated after 12 months) may/not support deriving a relative rate
we defined additional relative rates by CD4 count ($\le1$)
to reflect historical ART eligibility criteria \cite{NERCHA2018rep}.
We modelled viral suppression using a fixed rate for all subpopulations,
corresponding to an average of 4 months from ART initiation \cite{Mujugira2016}.
We modelled treatment failure / discontinuation with a single monotonically decreasing rate
applied to all subpopulations in the base case,
reflecting improving treatment success / retention over time \cite{NERCHA2014nsf}.
% SS: Curious why the decision to assume similar VS / retention given known barriers to VS
%     as well as potential underlying resistance issues. Perhaps just notable as a limitaiton?
% JK: Good point -- in next calibration we can explore <1 for FSW and/or clients too
Individuals with treatment failure / discontinuation could re-initiate ART at a fixed rate,
reflecting re-engagement in care or detection of treatment failure and initiation of alternative regimens.
We modelled rapid CD4 recovery during the first 4 months of ART,
% RK: Assume that this was just for mortality benefits and not transmission?
%     Latter not relevant on ART of course.
% JK: It is a bit of both I guess -- individuals on ART but not yet VS are modelled to have
%     25% (1, 67)% relative risk of transmission (reflecting large uncertainty)
%     on top of any reductions attributed to moving 'backwards' in HIV stage
%     (admittedly our 'HIV stage' strata conflate CD4 and VL)
%     The motivation for CD4 recovery is both for the reduced mortality and
%     reduced infectiousness when coming off ART.
%     Some more details in A.3.4.1 (HIV stage) & A.3.7.1 (ART effects)
followed by slower recovery while virally suppressed \cite{Battegay2006,Lawn2006,Gabillard2013}.
We modelled reduced HIV-attributable mortality among individuals on ART,
in addition to mortality benefits of CD4 recovery.
\paragraph{Calibration}
We calibrated the model to reflect
available data from Eswatini on HIV prevalence, HIV incidence, and ART cascade of care
% SM: suggest specifying if word count allows, which populations these calibration data refer to
% JK: there are a lot of different subpopulations (plus prevalence/incidence ratios)
%     maybe just highlight the tables which summarize these targets?
overall and stratified by subpopulation where possible
(Tables \ref{tab:targ.prev}--\ref{tab:targ.cascade})
\cite{SDHS2006,SHIMS1,SHIMS2,SHIMS3,Baral2014,EswKP2014,EswIBBS2022}
using an adapted version of Incremental Mixture Importance Sampling (IMIS) \cite{Raftery2010}.
Full methodology is given in Appendix~\ref{mod.cal},
while calibration results are given in Appendix~\ref{sr.cal}.
%===================================================================================================
\subsection{Scenarios \& Analysis}\label{art.meth.obj}
%---------------------------------------------------------------------------------------------------
\subsubsection{Objective~1: Influence of ART cascade differences between subpopulations}\label{art.meth.obj.1}
For Objective~\ref{obj:art.1},
We defined the \emph{base case} scenario to reflect
observed ART cascade scale-up in Eswatini, reaching \cashi by 2020 \cite{SHIMS3,AIDSinfo}.
% LW: Use actual exact numbers?
% JK: We calibrated to 95-95-95 even though the exact values might be slightly different,
%     those exact values were only estimated recently in SHIMS3
Next, we defined four \emph{counterfactual} scenarios in which
the overall population cascade was lower, reaching \casmd by 2020,
and where FSW, clients, both, or neither were disproportionately left behind.
% HM: Overall cascade?
% LW: Looks like not only vs lower, each cascade was lower?
% LW: I wonder if you can describe the four counterfactual scenarios as u did in Abstract first
%     (focused on cascade attainment as you call it: so it is more intuitive for readers).
%     And then u get into here RE how you operationalized it.
% JK: Great points -- have changed to 'cascade'
%     & introduced the 4 cases first before diving into the details as you suggest LW
In these counterfactual scenarios, we altered cascade attainment among
FSW, clients, and/or the remaining population by calibrating (``lower risk'')
constant relative scaling factors ``$R$'' to subpopulation-specific rates of:
diagnosis ($R_d \in [0,1]$),
treatment initiation ($R_t \in [0,1]$), and
treatment failure / discontinuation ($R_u \in [1,20]$).
When FSW and/or clients were left behind, we calibrated their $R$s such that
these populations attained approximately \caslo by 2020.
% SB: This sentence isnt immediately clear to me.
% JK: hopefully clarified?
By contrast, we calibrated $R$s for the remaining population such that
the Swati population \emph{overall} attained \casmd in all 4 counterfactual scenarios,
% SS: At least in SA, we've observed the third 90 to be problematic for FSW
%     so perhaps just worth noting this limitation of assumption in the discussion
% JK: Although I agree this is a limitation of our modelling assumptions,
%     and something we should explore in next calibration iteration,
%     I think this limitation would not directly undermine our findings,
%     so I might leave it out of the discussion?
thus ensuring that a consistent proportion of the population overall
attained viral suppression.
Table~\ref{tab:art.1.scen} summarizes these scenarios, while
Figure~\ref{fig:art.1.cascade} plots the modelled cascades over time.
When cascade rates among FSW and/or clients were unchanged from the base case,
the cascade these subpopulations attained could be lower than \cashi
due to subpopulation turnover and higher incidence.
All cascades continued to increase beyond 2020 due to assumed fixed rates of
diagnosis, treatment initiation, and treatment failure / discontinuation thereafter.
\begin{table}
  \centering
  \caption{Modelling scenarios for Objective~\ref{obj:art.1} defined by 2020 calibration targets}
  \label{tab:art.1.scen}
  \begin{tabular}{lCCCccc}
  \toprule
  & \multicolumn{3}{c}{ART cascade in 2020\tn{a}} 
  & \multicolumn{3}{c}{Re-scaled cascade rates\tn{b}} \\
  \cmidrule(rl){2-4}\cmidrule(rl){5-7}
  Scenario                                   &   FSW    & Clients  & Overall  & FSW & Clients & LR  \\
  \midrule
  \emph{Base Case}                           & 95-95-95 &    ---   & 95-95-95 & --- &   ---   & --- \\
  \emph{Leave Behind: FSW}                   & 40-60-80 &    ---   & 60-80-80 & \By &   \Bn   & \By \\
  \emph{Leave Behind: Clients}               &    ---   & 40-60-80 & 60-80-80 & \Bn &   \By   & \By \\
  \emph{Leave Behind: FSW \& Clients}        & 40-60-80 & 40-60-80 & 60-80-80 & \By &   \By   & \By \\
  \emph{Leave Behind: Neither}               &    ---   &    ---   & 60-80-80 & \Bn &   \Bn   & \By \\
  \bottomrule
\end{tabular}\floatfoot
\tnt[a]{Cascade: \% diagnosed among HIV+; \% on ART among diagnosed; \% virally suppressed among on ART};
\tnt[b]{Rates of: diagnosis; ART initiation; treatment failure}.
% TODO: explain counterfactual scenario re-fitting details?
\end{table}
% TODO: (~) create figure of these cascades like CROI poster
\par
We quantified ART prevention impacts via relative
% SM: nicely explained
% JK: thanks!
cumulative additional infections (CAI) and additional incidence rate (AIR)
% LW: Additional infections mentioned in Abstract refers to which?
%     Use consistent terminology in Abstract
% JK: CAI -- I've added 'cumulative' to Abstract
in the counterfactual scenarios ($k$) \vs the base case ($0$),
over multiple time horizons up to 2030, starting from $t_0 = 2000$:
\begin{equation}
  \txn{CAI, AIR}\,(t) = \frac{\Omega_{k}(t) - \Omega_{0}(t)}{\Omega_{0}(t)}
  ,\qquad \Omega(t) =
  \begin{cases}
    ~\int_{t_0}^{t}\!\Lambda(\tau)\,d\tau & \txn{CAI} \\
    ~\lambda(t) & \txn{AIR}
  \end{cases}
\end{equation} where:
$\Lambda$ denotes absolute numbers of infections per year, and
$\lambda$ denotes incidence rate per susceptible per year.
For each scenario, we computed these outcomes (CAI and AIR) for each model fit $j$,
and reported median (95\% confidence interval, CI) values across model fits, reflecting uncertainty.
% LW: Define CI in main text at its first occurence
% JK: done
\subsubsection{Objective~2: Conditions that maximize the influence of ART cascade differences}\label{art.meth.obj.2}
For Objective~\ref{obj:art.2}, we estimated via linear regression:
% LW: Details on the regression model? e.g., what type of regression model?
% JK: added 'linear' here, but full details are given below
the effects of lower ART cascade among FSW and clients on relative CAI and AIR,
% LW: "effect of" vs "association between". Also, specify risk groups {1: FSW, etc.} here
% JK: I think we can say effect b/c we generated synthetic data in a counterfactual framework
%     Also, changed certain subpopulations -> 'FSW and clients'
plus potential effect modification by epidemic conditions.
% SB: I think first time you used epidemic conditions. Could introduce the term earlier
%     and just indicate in abstract that studying which epidemic conditions are most impactful, etc.
% JK: Hmm, not sure where else to add? It was noted in the objectives (intro),
%     though I agree, it has not been defined until here...
The hypothesized causal effects are illustrated
as a directed acyclic graph in Figure~\ref{fig:art.2.dag}.
% SB: Nothing immediately follows this.
% JK: Moved that to next paragraph now:
\par
For this regression, we generated 10,000 synthetic samples as follows.
% SS: From where? From what distribution?
% JK: The whole paragraph describes the process to generate the synthetic data;
%     hopefully moving the 'as follows' here helps?
% LW: Is this para describing the IMIS? If so – I will call it out
% JK: This part is not IMIS (that was only for calibration)
%     though we do use LHS again for nice coverage of parameter space
We explored a wider range of counterfactual scenarios \vs Objective~\ref{obj:art.1}
by randomly sampling the relative rates for
diagnosis and treatment initiation $R_d, R_t \sim \opname{Beta}(\alpha=3.45,\beta=1.85)$
and treatment failure $R_u \sim \opname{Gamma}(\alpha=3.45,\beta=1.88)$
for each of: FSW, clients, and the remaining population (9~total values).
These sampling distributions had 95\%~CI: (0.25,~0.95) and (1.5,~15), respectively,
and were chosen to obtain cascades in 2020 spanning
approximately \mbox{60-60-90} through \mbox{90-90-95} (Figure~\ref{fig:art.2.cascade}). % MAN
% LW: Can u clarify how these ranged are chosen?
% JK: Trying to reflect the range of cascades observed/expected by 2020 across SSA
For each of $N_f = 1000$ model fits,
we generated $N_k = 10$ counterfactual scenarios per fit
using Latin hypercube sampling \cite{Stein1987} of $R$s,
% LW: Can add a citation for LHS
% JK: done
yielding $N_f N_k = {}$10,000 total counterfactual samples for the regression.
\begin{figure}
  \centering\includegraphics[scale=1]{art.2.dag}
  \caption{Directed acyclic graph (DAG) for inferring
    the epidemic conditions under which
    differential viral suppression across subpopulations matters most}
  \label{fig:art.2.dag}
  \floatfoot{
    $Y$: cumulative additional infections (CAI) or additional incidence rate (AIR) by 2030;
    $D$: difference in population-overall viral non-suppression
      in counterfactual \vs base case scenario;
    $d_i$: difference in subpopulation-$i$-specific viral non-suppression
      \vs population overall within counterfactual scenario;
    $C_j$: epidemic conditions (effect modifiers of $d_i$).}
\end{figure}
\par
For each of these 10,000 samples, we defined
relative CAI and AIR by 2030 \vs the base case, as in Objective~\ref{obj:art.1}.
% SM: one thing we may run into as an acroynm issue is that
%     CAI is often used in HIV literature to mean condomless anal intercourse.
%     I think that is not an issue in this paper,
%     but just a heads-up editors may ask us to remove acronym here
%     (in previous experience with journals which have to reduce number of acronyms)
% JK: Ah, good to know! Let's see what they say...
For each sample, we further defined
$U_{fki}$ for subpopulations $i \in \{\tdef{1}{FSW}, \tdef{2}{clients}, \tdef{*}{overall}\}$
as the proportions \emph{not} virally suppressed among those living with HIV by 2020,
% SM: unsupressed always sounded a bit funny to me and am not super used to it yet lol,
%     but no need to change this if we are seeing it more and more in HIV papers
% SB: This isnt a word so will need to define how being used here (assume just not suppressed).
%     But why can't framing be viral suppression?
% SS: maybe "detectable viral load" (> XX copies/ml)
% JK: I've changed 'unsuppression' -> 'non-suppression', which is maybe slightly better?
%     My reservation about 'suppression' vs 'non-suppression' framing is that it flips the effect directions
%     such that the effects become less intuitive (to me anyway).
%     Since all counterfactual scenarios are expected to have more infections vs base case (95-95-95),
%     we now talk about higher non-suppression -> more additional infections; whereas
%     we would instead have higher suppression -> fewer additional infections (but still more vs base case)
%     Maybe the above is not bad, but then interpreting the influence of epidemic conditions is also tricky:
%     \beta_{ij} would represent the relative decrease in additional infections
%     due to viral suppression among subpopulation i in the context of condition C_j ---
%     I feel the current meaning of \beta_{ij} as 'relative increase in additional infections
%     due to non-suppression among subpop i in the context of C_j', is more intuitive?
reflecting a summary measure of ART cascade gaps.
Using $U_{fki}$, we defined the main regression predictors as:
$D_{fk} = U_{fk*} - U_{f0*} > 0$, reflecting differences in
\emph{population overall} viral non-suppression in sample $k \in [1,10]$
\vs the base case (denoted $k = 0$); and
$d_{fki} = U_{fki} - U_{fk*} \lessgtr 0$, reflecting differences in
\emph{subpopulation-$i$-specific} viral non-suppression in sample $k$
\vs the population overall in sample $k$ --- \ie viral non-suppression inequalities.
\par
Next, we defined the following measures of epidemic conditions ($C_{fj}$) related to sex work,
as hypothesized modifiers of the effect of unequal non-suppression on relative CAI and RAI:
FSW and client population sizes (\% of population overall);
average rate of turnover among FSW and clients (per year, reciprocal of duration selling / buying sex); and
HIV prevalence ratios in the year 2005 among FSW \vs other women, and among clients \vs other men.
For these measures, we combined FSW at higher and lower risk,
and likewise for clients at higher and lower risk.
We used HIV prevalence ratios in 2005 to reflect summary measures of risk heterogeneity prior to ART,
% SM: specifiy we mean prevalence ratios between populations/groups?
% JK: HIV prevalence ratios are defined just above (1st sentence in this para)
%     not sure what other prevalence ratios we would mean?
rather than including all risk factors from the transmission model,
% SM: by modelled we mean in the transmission model yes?
%     may help to distinguish from the statistical model when we talk about both together
% JK: ah, good point
which could lead to overfitting and improper inference due to effect mediation.
\par
Finally, we defined a general linear model for each outcome (CAI, AIR) as:
% SM: I know you write the model in the order your conduct the analysis.
%     But you may have to write the Methods in a way that help readers get through the bigger picture first; then the details.
%     e.g., you can note generalized linear model early in para 1 when you mention regression analysis;
%     Then you talk about what is your outcome, exposure, other covariate of interest (here is the potential effect modifier of interest);
%     Then you go into definitions of each.
%     Then you go into how you generated/simulated the dataset via sampling.
%     To the same note, once you sort through the bigger picture parts and their order:
%     u can then keep some specifcs/details in appendix.
% JK: I've added a bit more big picture framing earlier per the suggestions above
%     I think the para 1 now gives the big picture, in simple terms (i.e. not specifying conditions)
%     I'm a bit hesitant to add another layer of detail in between para 1 and full details
%     as it might just add repetition / confusion?
\begin{equation}\label{eq:art.2.glm}
  \txn{CAI, AIR} = \beta_0\,D
                 + \sum_i \beta_i\,d_i
                 + \sum_{ij} \beta_{ij}\,d_i\,C_j
\end{equation}
such that each outcome was modelled as a sum of the effects of:
differential population-level non-suppression in the counterfactual \vs the base scenario ($D$);
% SM: any other terms we can use instead of "unsuppression"?
%     and thinking re: consitency with terminology with abstract/intro, discussion later, etc.
% JK: see above re. unsupression -> non-suppression
differential non-suppression among FSW and clients
\vs the population overall within the counterfactual scenario ($d_i$); and
effect modification of $d_i$ by epidemic conditions ($C_j$).
The model does not include an intercept because if $D = d_i = 0$,
then we expect $\txn{CAI} = \txn{AIR} = 0$.
We fitted this model for each outcome using generalized estimating equations \cite{Hojsgaard2006}
to control for repeated use of each model fit $f$.
We standardized all model variables ($D$, $d_i$, $C_j$) via
$\hat{x} = (x - \txn{mean}(x)) / \txn{SD}(x)$
to avoid issues of different variable scales and collinearity in interaction terms.
% SM: nice
This standardization does not imply that
regression coefficient magnitudes can be compared to indicate variable ``importance'',
because the standardization applied to each variable is driven
by the variance before standardization
--- in this case reflecting arbitrary ranges ($D$, $d_i$) or uncertainty in calibration ($C_j$).%
\footnote{We verified that results were qualitatively unchanged using
  $\hat{x} = (x - \txn{median}(x)) / \txn{IQR}(x)$.
  For further discussion on interpretation of standardized regression coefficients,
  see also: \hreftt{stats.stackexchange.com/questions/29781} and links therein.}
Rather, effect sizes can be interpreted as:
the expected change in outcome per standard deviation change in the variable.
