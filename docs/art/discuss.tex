\section{Discussion}\label{art.disc} %SM: this is an excellent start, and I think almost all content there; and I really like your structuring too! I think would benefit from some more work together (very sorry again about my delay with this!!) re: a bit more re: concrete examples (mostly 2 paragraphs) - lets chat Jan 8, and before that- see what you think re: my comments/suggestions. I think with one more round of review/edits with me re: the discussion section and we'll be ready to submit! 
By modeling the successful scale-up of ART in Eswatini, we 
estimated the consequence of inequalities in the HIV cascade and intersections of risk heterogeneity on
the popualtion-level prevention impacts of ART.
We found that ART scale-up that ``leaves behind'' higher risk groups,
such as female sex workers (FSW) and their clients, woudl have led to  
X-X% more HIV infections,
even for the same population-overall coverage.
We also found that the transmission impact of
leaving behind FSW and/or clients was largely determined by
characteristics of the client population, including:
population size, turnover, and relative HIV prevalence.
\par %SM: this para could  benefit from a bit more substance as it is too general for a paper that is focused on Eswatini. Tell reader more about Eswatini program here, etc. - esp for journal like Lancet HIV
Available data suggest that
Eswatini has recently surpassed 95-95-95 for the population overall \cite{SHIMS3}.
Among FSW living with HIV, the proportion diagnosed and the proportion on ART was at X% and X% in 202X \cite{EswIBBS2022}, %SM: give the latest empirical estimates here...
though data on viral suppression are lacking.
These achievements coincide with rapid reductions in HIV incidence in recent years
\cite{SHIMS1,SHIMS2,SHIMS3} (Figure~\ref{fig:fit.inc}).
In the context of MaxART, HIV testing, treatment, and adherence programs tailored for FSW included....
%SM: talk about the FSW program re: testing, ART scale up, etc. here. I would be specific here and not general re: inspiration, etc. (that reads a bit editorializiing perhaps on our part) -  but rather focus on explaining how high ART coverage and VLS was achieved among FSW. 

However, differences in the the HIV cascade persisted among men \vs women - a pattern consistent with that of countries across Sub-Saharan Africa %SM: worse in men yes? woudl say that
 \cite{SHIMS1,SHIMS2,SHIMS3}. %SM: not clear what 'survey coverage' means. also, its not clear what "familiar" means to general reader - specifiy / tell reader that this is a pattern seen across SSA?
Such differences should continue to be examined and addressed, %SM: a bit too vague/generic statement - can we say something about clients here?  
since their drivers may overlap with drivers of HIV risk \cite{Akullian2017,Camlin2019}. %SM: whose drivers?
\par
Our findings are likely generalizeable to other epidemic contexts. %SM: other or similar epidemic contexts? 
HIV prevalence ratios between key populations and the population overall
are relatively low in Eswatini \vs elsewhere \cite{Baral2012,Hessou2019};
thus, the transmission impact of cascade gaps among key populations in other contexts
would likely be even greater than we found for Eswatini.
Moreover, as HIV incidence declines in many settings,
epidemics may become re-concentrated among key populations \cite{Brown2019,Garnett2021},
further magnifying the transmission impact of cascade disparities.
\par %SM: suggest revising this paragraph a lot. most of this seems like introduction material, and rather - perhaps here the message is re: demonstrating the importance of X, confirmiing prior findings but adding new model-based evidence -- value of doing this retrospectively, etc.? also, could say how this relate to prior modeling literature on prevention gaps at the intersection of heightended risks in general, beyond FSW, etc. to consider clients, MSM,  ... i.e. as a general principle
To our knowledge, our study is the first to explore the transmission impact of
heterogeneity in ART coverage across risk groups,
within consistent population-overall coverage,
though \citet{Maheu-Giroux2017art} examined the impact of
``leaving behind'' key populations in C\^{o}te d'Ivoire. %SM: lets be more precise - what did we add / confirm re: prior studies? its not clear from this sentence how our findings differ or relate to that of MMG's paper
In our previous review of mathematical modelling of ART scale-up in Sub-Saharan Africa,  %SM: feels odd to read here - does this belong in introduction instead? 
we found that few modeliing studies have considered any cascade differences by risk group, %SM: speciffy / be more precicse re; modeling studies. b/c we talk about differnet studies in the discussion..
but that such differences likely mediate ART prevention impacts \cite{Knight2022sr}.
Cascade gaps have been observed among men \vs women \cite{Quinn2019,Green2020}, %SM: this all seems like it belongs in introduction. 
younger \vs older populations \cite{Green2020,Lebelonyane2021},
key populations \vs the population overall \cite{Hakim2018},
and within key populations themselves \cite{Mayanja2018,Jaffer2022}.
Moreover, unmeasured cascades  %SM: already said in introduction? 
--- such as among populations who have not been reached by programs and interventions ---
are likely lowest \cite{Hakim2018,Boothe2021}.
Consistent integration of these data going forward could
improve the quality of model-based evidence for HIV resource prioritization.
\par
Global ART scale-up has many benefits, including for
individual-level health outcomes \cite{Gabillard2013,Lundgren2015init},
prevention in serodiscordant relationships \cite{Cohen2016},
and contributing to population-level incidence declines \cite{Havlir2020}.
However, efforts to maximize cascade coverage should not overlook
populations that may be harder to reach,
where barriers to engagement in HIV care often intersect with drivers of HIV risk
\cite{Wanyenze2016,Schwartz2017,Schmidt-Sane2022,Camlin2019,Baral2019}.
Such populations can be reached effectively through
tailored services to meet their unique needs \cite{Ehrenkranz2019},
services which can be designed and refined with ongoing community engagement
\cite{Chikwari2018,Mlambo2019,Comins2022}.
As we have shown, an equity-focused approach to ART scale-up can maximize prevention impacts,
and accelerate the end of the HIV epidemic.
\par
Our study has three major strengths.  %SM: very nice
First, drawing on our conceptual framework for risk heterogeneity \cite[Table~1]{Knight2022sr}
and multiple sources of context-specific data
\cite{SDHS2006,SHIMS1,SHIMS2,Baral2014,EswKP2014,EswIBBS2022},
we captured several dimensions of risk heterogeneity, including:
heterosexual anal sex,
four types of sexual partnerships,
sub-stratification of FSW and clients into higher and lower risk,
and risk group turnover
(Table~\ref{tab:art.het}, Appendix~\ref{mod}).
Second, whereas most modelling studies of ART scale-up
project hypothetical future scenarios which may not be achievable,
our base case scenario reflects observed scale-up in Eswatini.
Finally, our analytic approach to Objective \ref{obj:art.2},
in which epidemic conditions are conceptualized as potential effect modifiers,
represents a unique methodological contribution to the HIV modelling literature.
\par
Our study has three main limitations.
First, we only considered heterosexual HIV transmission in Eswatini,
and mainly explored risk heterogeneity related to sex work.
However, our findings would likely generalize
to other transmission networks and determinants of risk heterogeneity,
including risk groups not always recognized as key populations,
such as mobile populations and young women \cite{Camlin2019,Cheuk2020}.
Second, our model structure did not include age.
However, our calibrated model fits
the available age-aggregated data well (Appendix~\ref{sr.cal}),
and the qualitative interpretation of our findings
is unlikely to be changed by adding age stratification.
Finally, we did not consider transmitted drug resistance.
Drug resistance is more likely to emerge
in the context of barriers to viral suppression \cite{Pham2014};
thus, cascade gaps among those at higher risk
would likely accelerate emergence of transmitted drug resistance, and amplify impacts of such gaps.
\par
In conclusion, HIV prevention efforts should be rooted in  
context-specific understandings of prevention gaps. %SM: did we talk about prevention gaps  before? 
In the ``treatment as prevention'' era, prevention gaps include cascade gaps. %SM: did we mention treatment as prevention before? feels like new terminoloogies being introduced here..?
Thus, differential cascades within and between populations at higher risk of HIV
must be described, modelled, and ultimately addressed
to fully realize the anticipated prevention impacts of ART.
\enlargethispage{2ex} % TEMP
