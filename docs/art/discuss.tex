\section{Discussion}\label{art.disc}
We sought to explore how inequalities in the ART cascade
that intersect with HIV risk heterogeneity
may influence the model-estimated prevention impacts of ART.
In our applied analysis of Eswatini, we found that
slower cascade scale-up that left behind female sex workers (FSW) and their clients
could have resulted in 50--167\% more infections by 2030 \vs slower scale-up alone.
We also found that the impact of
leaving behind FSW and/or clients was largely determined by
characteristics of the client population, including:
population size, turnover, and relative HIV prevalence.
\par
Eswatini has recently surpassed 95-95-95 for the population overall \cite{SHIMS3}.
These targets were achieved through
numerous initiatives coordinated across sectors,
including those led by the \emph{MaxART} program \cite{MaxART2}.
Multiple stakeholders, including people living with HIV, healthcare providers,
% SM: i like this, and this definitely needs citations - is it the MaxART2 citation?
% JK: yep!
traditional and religious leaders, community groups, and researchers
were engaged via multiple channels, such as
Technical Working Groups, Community Advisory Boards,
and specific meetings for prioritized groups (men and adolescents) \cite{MaxART2}.
% BS: I am not sure about this sentense and we might need to check the sourse.
%     The most common engagements at program level are
%     technical working groups (TWGs) for different HIV thematic areas.
% JK: see below
Drawing on this engagement and
social science research to understand barriers to care,
cascade services were comprehensively strengthened via investments in
training, infrastructure, anti-stigma communication, demand creation, and monitoring~%
\cite{MaxART2}.
% SM: love this. and definitely needs citations
% JK: Yeah -- the MaxART programs & reports were/are awesome! Here is a link:
%     https://aidsfonds.org/work/maxart-phase-ii-early-access-to-art-for-all/resource
%     I would cite both reports, but we are already 7 refs over the limit...
% LM: Is this describing how the MaxART program engaged or the larger HIV program.
%     Clarity on this sentence should be made.
%     In support of Bheki most of the programmatic, coordination decisions,
%     including in studies (in addition to study specific meetings)
%     includes effective TWGs that include all critical stakeholders.
%     It might also help to note that stigma and discrimination sessions
%     were multisectoral, and trainings were as well.
%     I think this is one of the successes of the Eswatini program
%     how comprehensive it was to address issues.
% JK: We were hoping to describe efforts both in MaxART and others by MOH in general
%     I've edited to indicate that MaxART was just group of initiatives
%     and specifically highlight the TWGs.
\par
Among FSW living with HIV in Eswatini, data suggest that
88\% were diagnosed and 86\% were on ART in 2021
(\ie ART coverage was 98\% among those diagnosed) \cite{EswIBBS2022}.
% BS: I could not understand the 86%.
% LM: It took me a while but if I got it it was that
%     of the 88% that self reported to be diagnosed HIV positive,
%     86% of those, or 98%, are on ART. But it took me some time to get it.
% JK: hopefully clarified with edits here?
Data on viral suppression among FSW were lacking,
and so assumed to be similar to other women in the simulated epidemic
(see \sref{mod.par.cascade} for more details).
% SM: or was it equivalent to women overall?
% JK: we allowed 1-1.5x rate of ART failure among FSW vs women overall,
%     plus turnover dynamics reduced a bit too
%     see: \ref{fig:fit.vls.c} -- i.e. Figure B.5 d)
Although lower than 95-95-95, this cascade among FSW living with HIV
is higher than in many other regions \cite{Schwartz2017,Hakim2018}.
% SM: cite data from other regions/countries
% JK: done
Strong programs are required to attain high cascades among FSW,
considering that women enter and exit sex work (turnover) and
likely experience highest risk of HIV acquisition during sex work.
% LM: I would think with high incidence we would see greater progress against the 95-95-95.
%     I don't believe the modelling can show this but I would think that
%     it is impressive based on the continual barriers, or perceived barriers,
%     to access health services that make it hard to find FSWs for access to
%     testing services, if they don' t know their status,
%     and access to treatment and support to stay on treatment.
% JK: Re. incidence: I think with a higher rate of new infections,
%     it would be harder to 'keep up' with an increasing number of PLHIV ?
%     i.e. if the 'denominator' in 95% = (% diagnosed / PLHIV) keeps growing,
%     it's harder to reach / maintain 95% vs a situation where PLHIV is not growing.
%     We kind of explored this (indirectly) in a previous study on turnover.
%     But hopefully we also speak to the other reason it is impressive re. barriers below?
That is, programs must ensure higher rates of HIV testing, ART initiation, and retention
among FSW \vs other women to achieve similar cascades.
% BS: Should we mention this testing should target those not knowing their status?
For example, we inferred that rates of HIV testing in 2016 was 80--227\% higher
among FSW \vs other women to reproduce observed cascade data during model calibration.
% SM: I would include these numbers (in data-informed modeling,
%     always good to include what the work could show, to support the statements)
% JK: This is a little tricky from the code, since we allowed the relative rate of testing
%     among FSW vs other women to increase over time -- reflecting KP program scale-up
%     hence specifying 'in 2016'.
\par % could keep / remove this par break
In Eswatini, programs for key populations include safe access to tailored services via
drop-in centers (locally known as TRUE), mobile outreach, venue-based, and one-on-one options~%
\cite{EswIBBS2022}.
% SM: cite
% JK: done
Health and clinical services are also integrated with
efforts to reduce structural vulnerabilities, including
experiences of harassment, violence, and fear of seeking healthcare,
through community empowerment, psycho-social and legal supports, and
sensitization and training for police and healthcare workers \cite{EswIBBS2022}.
% SM: very nice
These programs have been designed and refined with ongoing community leadership and engagement,
allowing them to better meet the specific needs of key populations,
for whom barriers to engagement in HIV care often intersect with drivers of HIV risk,
including economic insecurity, mobility, stigma, discrimination, and criminalization
\cite{Lancaster2016sr,Wanyenze2016,Schwartz2017,Schmidt-Sane2022,Camlin2019,Baral2019}.
% LM: This fits more around why progress for KPs is impressive b/c this is true.
% SM: I recognize a long paragraph, but would consider combining
%     b/c sayiing similear infomation and the 2nd one
%     read a bit repetitive after reading the first...
%     where the first was being specific, and the 2nd was being more generic/general.
%     I would combine into 1 para and close para with a strong statement about equity like you have
% JK: love it!
% SM: this sentence below repeats much of same "content" as sentence around KP programmes in Eswatini
%     - can we combine it?
% JK: done
Our data-informed modeling of cascade scale-up in Eswatini confirms that
such an equity-focused approach to ART cascade scale-up
% SM: lets avoid saying "we showed" since it feels like this is
%     something many others have shown/known re: principles of equity :)
%     even if not specifically in the case of HIV cascade.
% JK: love this too! I did change to "our data-informed..." from "the data-informed..."
%     since I was worried it might not be clear which modelling we are talking about here
can maximize prevention impacts, and accelerate overall reductions in HIV incidence.
% SM: we have not defined what is meant by "end of AIDS epidemic".
%     so lets avoid that term unless referrinig to a specific UNAIDS report title, etc.
% JK: gotcha -- agree!
\par
Our study highlights the importance of reaching both FSW and their clients,
echoing recent modelling studies of South Africa and Cameroon \cite{Stone2021,Silhol2024}.
These studies found that gaps in HIV prevention and treatment for clients
were among the largest contributors to onward transmission in recent years.
Such findings reiterate the need for improved data on both FSW and clients,
including estimates of population size, sexual behaviour, and ART cascade attainment.
These estimates may be difficult to obtain
because individuals are unlikely to report buying or selling sex in population-level surveys
due to stigma and criminalization \cite{Behanzin2013} and
because many clients are highly mobile
(including transient seasonal/occupational migration) \cite{Camlin2019}.
Thus, innovative study designs, bias adjustments, and services may be needed
to understand and meet clients' needs.
\par
While numerous modelling studies have examined
the potential prevention impacts of ART cascade scale-up
\cite{Knight2022sr} (Appendix~\ref{sr.ric}),
our study is the first to explore the impact of
inequalities in ART cascade across subpopulations
with consistent population overall cascade across scenarios.
Similar work by \citet{Marukutira2020} illustrated the limited impact of
achieving 95-95-95 for only citizens and not immigrants in Botswana,
while \citet{Maheu-Giroux2019cost} illustrated the high cost-effectiveness of
prioritizing key populations (including clients) for ART in C\^{o}te d’Ivoire.
Indeed, our findings are likely generalizable to other epidemic contexts.
HIV prevalence ratios between key populations and the population overall
are relatively low in Eswatini \vs elsewhere;
thus, the impact of cascade inequalities among key populations in other contexts
would likely be even greater than we found for Eswatini.
Moreover, as HIV incidence declines in many settings,
transmissions may become concentrated among key populations \cite{Brown2019},
further magnifying the impact of cascade inequalities.
\par
A primary strength of our analysis is the use of
observed ART cascade scale-up to 95-95-95 in Eswatini as the base case,
with plausible cascade inequalities explored in counterfactual scenarios.
As noted above, the available data suggest that Eswatini has
minimized cascade inequalities which persist elsewhere \cite{Hakim2018}.
Thus, our counterfactual scenarios directly estimate
the consequences of failing to address these inequalities.
Second, drawing on our conceptual framework for risk heterogeneity \cite[Table~1]{Knight2022sr}
and multiple sources of context-specific data
\cite{SDHS2006,SHIMS1,SHIMS2,Baral2014,EswKP2014,EswIBBS2022},
we captured several dimensions of risk heterogeneity, including:
heterosexual anal sex,
four types of sexual partnerships,
sub-stratification of FSW and clients into higher and lower risk strata,
and subpopulation turnover
(Table~\ref{tab:art.het}, Appendix~\ref{mod}).
Accurate modelling of risk heterogeneity
has been shown to mediate model-estimated ART prevention impacts \cite{Hontelez2013},
and is especially important when considering differential ART scale-up across subpopulations.
Finally, our analytic approach to Objective \ref{obj:art.2},
in which epidemic conditions are conceptualized as potential effect modifiers
represents a unique methodological contribution to the HIV modelling literature.
\par
Our study also has limitations.
First, we did not model pre-exposure prophylaxis (PrEP).
However, our analyses focus on the time period
prior to widespread PrEP availability in Eswatini.
Second, we did not consider transmitted drug resistance (TDR).
However, drug resistance is more likely to emerge
in the context of barriers to viral suppression;
thus, lower cascade among those at higher risk
would likely accelerate emergence of transmitted drug resistance,
and thereby magnify our findings.
Finally, our model structure did not include age,
and we only considered heterosexual HIV transmission in Eswatini.
Future work can explore adaptation of the model to consider
PrEP, TDR, age stratification, and additional modes of HIV transmission.
While the magnitude of our results may change with such adaptations,
we do not expect that the qualitative interpretation would change.
In fact, our findings would likely generalize
to other transmission networks and determinants of risk heterogeneity,
including other key populations and subpopulations such as
highly mobile populations and young women \cite{Camlin2019,Cheuk2020}.
\par\pagebreak % TEMP
In conclusion, the HIV response must remain rooted in
context-specific understandings of inequalities in HIV risk and in access to HIV services,
which often stem from common upstream factors.
Thus, differences in ART cascade within and between subpopulations at higher risk of HIV
must be monitored, characterized, and addressed
to fully realize the anticipated benefits of ART
at both the individual and population levels.
