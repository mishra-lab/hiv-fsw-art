\section{Discussion}\label{art.disc}
% SM: this is an excellent start, and I think almost all content there;
%     and I really like your structuring too! I think would benefit from some more work together
%     (very sorry again about my delay with this!!) re: a bit more re: concrete examples (mostly 2 paragraphs)
%     lets chat Jan 8, and before that- see what you think re: my comments/suggestions.
%     I think with one more round of review/edits with me re: the discussion section and we'll be ready to submit!
We sought to explore how inequalities in the ART cascade
that intersect with HIV risk heterogeneity
may influence the model-estimated prevention impacts of ART.
% LW: Consistency again: ART coverage vs. cascade vs...
% JK: thanks!
We found that cascade scale-up that leaves behind subpopulations at higher HIV risk,
such as female sex workers (FSW) and their clients,
could lead to substantially more HIV infections,
% JK: Re. removed quoted % more infections from results:
%     although I appreciate trying to be precise, I feel our analysis is more illustrative
%     because the cascades examined in counterfactual scenarios are a bit arbitrary (and not re-stated here).
%     So, I'd rather make a more general statement?
even for the same population overall ART scale-up.
We also found that the impact of
% SM: Prevention or transmission? I think critical to be consistent with language.
% JK: Usually we say prevention impact of ART, since ART is preventative,
%     but here it is the impact of leaving behind {subpopulations},
%     so it feels funny to read prevention impact of leaving behind ...
%     I know we discussed leaving 'transmission impact' vs just 'impact' here,
%     but I personally feel the term 'transmission impact' is not really defined so could be confusing.
%     Given the 2 prior sentences, I feel it is clear we are talking about impact re. infections?
leaving behind FSW and/or clients was largely determined by
characteristics of the client population, including:
population size, turnover, and relative HIV prevalence.
% LW: I feel the current Discussion lack contents talking about the implications of the Obj 2 findings:
%     E.g., impliations on collecting and estimating data on client population size; client HIV prevalence estimates etc.
% JK: Good point --- I've added below in the para contextualizing wrt modelling literature
\par
% SM: this para could benefit from a bit more substance as it is too general for a paper that is focused on Eswatini.
%     Tell reader more about Eswatini program here, etc. - esp for journal like Lancet HIV
Available data suggest that
Eswatini has recently surpassed 95-95-95 for the population overall \cite{SHIMS3}.
Among Swati FSW living with HIV, self-reported proportions diagnosed and on ART
were estimated to be 88\% and 86\% (98\% among diagnosed) in 2021 \cite{EswIBBS2022},
% SM: give the latest empirical estimates here...
% JK: done
though data on viral suppression are lacking.
% LW: but u mention data on VS are lacking?
% SB: Wasn't this based on self-reported treatment? I think really critical that this be specifically said here
%     ie, that treatment appears good but there are large disconnects between self-reported ART and viral suppression
%     for a number of reasons including social desirability, resistance, etc.
% JK: done
% SS: Again noting, however, that we've proposed higher VS attainment than for first two steps
%     but here we seem to suggest that potentially the opposite may be true
% JK: We did allow a higher treatment failure/discontinuation rate for FSW,
%     in addition to potentially lower VS due to turnover effects,
%     so the modelled VS could be lower (Figure~\ref{fig:fit.cascade}) than other subpopulations
These achievements coincide with rapid reductions in HIV incidence in recent years
\cite{SHIMS1,SHIMS2,SHIMS3} (Figure~\ref{fig:fit.inc}).
Efforts to scale-up ART in Eswatini have been multi-faceted, including \dots [TODO]
% --------------------------------------------------------------------------------------------------
% SM: talk about the FSW program re: testing, ART scale up, etc. here.
%     I would be specific here and not general re: inspiration, etc. (that reads a bit editorializiing perhaps on our part)
%     but rather focus on explaining how high ART coverage and VLS was achieved among FSW.
% JK: TODO (soliciting co-author input)
% JK: might remove the text below depending on how it links with the inputs above.
% SM: worse in men yes? would say that
% LW: familiar?
% JK: edited for clarity.
% SM: not clear what 'survey coverage' means. also, its not clear what "familiar" means to general reader
%     specifiy / tell reader that this is a pattern seen across SSA?
% JK: edited to clarify: 'survey participation'
% SM: a bit too vague/generic statement - can we say something about clients here?
% SM: whose drivers?
% SB: Feels like getting in the weeds pretty quickly here.
%     I would lay out overall messages and then can get into specifics.
% JK: have removed the text above.
% --------------------------------------------------------------------------------------------------
\par
Our findings are likely generalizable to other epidemic contexts.
% SM: other or similar epidemic contexts?
% JK: I am comfortable saying 'other', not just 'similar'
%     but it might be stretching too far without showing any evidence
HIV prevalence ratios between key populations and the population overall
are relatively low in Eswatini \vs elsewhere \cite{Baral2012,Hessou2019};
thus, the impact of cascade inequalities among key populations in other contexts
would likely be even greater than we found for Eswatini.
Moreover, as HIV incidence declines in many settings,
epidemics may become re-concentrated among key populations \cite{Brown2019,Garnett2021},
further magnifying the impact of cascade inequalities.
% LW: Saw Stef's edits using wording like inequity; just to make sure consistency
% JK: thanks -- got it!
\par
% SM: suggest revising this paragraph a lot. most of this seems like introduction material, and rather
%     perhaps here the message is re: demonstrating the importance of X,
%     confirming prior findings but adding new model-based evidence -- value of doing this retrospectively, etc.?
%     also, could say how this relate to prior modeling literature on prevention gaps
%     at the intersection of heightended risks in general, beyond FSW, etc. to consider clients, MSM,
%     ... i.e. as a general principle
% JK: Have revised to focus on contextualizing with other modelling
%     & highlighting the importance of clients --- what do you think?
%     Only thing I've omitted from suggested is the 'value of doing this retrospectively',
%     since this is noted below in the 'strengths' para
While numerous modelling studies have examined
the potential prevention impacts of ART cascade scale-up
\cite{Knight2022sr} (Appendix~\ref{sr.ric}),
our study is, to our knowledge, the first to explore the impact of
inequalities in ART cascade across subpopulations,
with consistent population overall cascade across scenarios.
Our findings nevertheless parallel those of
\citet{Maheu-Giroux2017art} and \citet{Marukutira2020},
who highlight the limited impact of achieving 90-90-90+
\emph{only} among subpopulations at lower risk of HIV.
% --- specifically, if key populations, including clients of FSW, and immigrants are left behind.
% JK: not sure whether the above details re. populations explored in 2 studies are needed
%     the phrasing is a bit awkward, and maybe not needed.
The importance of reaching clients of FSW in particular
has also been emphasized in recent models \cite{Stone2021,Silhol2024},
with the added relevance that highly mobile men are more likely to be both clients,
and missing from population-level estimates of ART cascade attainment
\cite{Akullian2017,Camlin2019}.
Such findings reiterate the urgent need for data to characterize
client populations alongside traditional key populations,
including estimates of population size, HIV burden, and unmet needs.
% SM: lets be more precise - what did we add / confirm re: prior studies?
%     its not clear from this sentence how our findings differ or relate to that of MMG's paper
% JK: Have clarified how ours is unique, re. consistent population overall cascade across scenarios
%     and have now removed the text referenced in the comments below.
% SM: feels odd to read here - does this belong in introduction instead?
% SM: specify / be more precicse re; modeling studies. b/c we talk about differnet studies in the discussion...
% LW: Would use more generic term than "mediate"
% SM: this all seems like it belongs in introduction.
% LW: Consistency in terminology: cascade gaps vs. cascade disparities
% JK: done
% SM: already said in introduction?
\par
Global ART scale-up has many benefits, including for
individual-level health outcomes \cite{Gabillard2013,Lundgren2015init},
prevention in serodifferent relationships \cite{Cohen2016},
% SS: Previously said serodifferent
% JK: done
and contributing to population-level incidence declines \cite{Havlir2020}.
However, efforts to maximize cascade coverage should not overlook marginalized populations,
% SS: "not be as routinely reached"
% JK: changed 'populations that may be harder to reach' -> 'marginalized populations'
for whom barriers to engagement in HIV care often intersect with drivers of HIV risk,
including economic insecurity, mobility, stigma, discrimination, and criminalization
\cite{Wanyenze2016,Schwartz2017,Schmidt-Sane2022,Camlin2019,Baral2019}.
Such populations can be reached effectively through
tailored services to meet their specific needs \cite{Ehrenkranz2019},
services which can be designed and refined with ongoing community leadership and engagement
% SB: And may want to say community-leadership rather than just engagement.
% JK: great point!
\cite{Chikwari2018,Mlambo2019,Comins2022}.
As we have shown, an equity-focused approach to ART cascade scale-up
can maximize prevention impacts, and accelerate the end of the HIV epidemic.
\par
Our study has three major strengths.
% SM: very nice
First, drawing on our conceptual framework for risk heterogeneity \cite[Table~1]{Knight2022sr}
and multiple sources of context-specific data
\cite{SDHS2006,SHIMS1,SHIMS2,Baral2014,EswKP2014,EswIBBS2022},
we captured several dimensions of risk heterogeneity, including:
heterosexual anal sex,
% LW: Was anal sex explicitly mentioned in methods. If not - please add
% JK: Yes, but only in Table~\ref{tab:art.het} (1)
four types of sexual partnerships,
sub-stratification of FSW and clients into higher and lower risk strata,
and subpopulation turnover
(Table~\ref{tab:art.het}, Appendix~\ref{mod}).
Second, whereas most modelling studies of ART cascade scale-up
project hypothetical future scenarios which may not be achievable,
our base case scenario reflects observed scale-up in Eswatini.
% LW: Elaborate why this is a strength in relation to ur research question
% JK: I think it is a bit self-evident?
Finally, our analytic approach to Objective \ref{obj:art.2},
in which epidemic conditions are conceptualized as potential effect modifiers
% and regression models are employed to examine influence of key parameters on projected population-level impact
% JK: I think to remove the suggested above, b/c
%     prior approaches to sensitivity analysis (PRCC) essentially use linear models
%     just in a causal framework to try to isolate effects
represents a unique methodological contribution to the HIV modelling literature.
\par
% LW: mention did not PrEP
% JK: done (below)
Our study has four main limitations.
First, we only considered heterosexual HIV transmission in Eswatini,
and mainly explored risk heterogeneity related to sex work.
% RK: A sentence or two about barriers to care and ART rollout among MSM in SSA would fit well, if space allows
% JK: Although did not call out MSM specifically
%     I've tried to add more specific barriers outlined above
%     (and focused a bit on clients due to our focus and findings)
However, our findings would likely generalize
to other transmission networks and determinants of risk heterogeneity,
including subpopulations not always recognized as key populations,
such as mobile populations and young women \cite{Camlin2019,Cheuk2020}.
Second, our model structure did not include age.
However, our calibrated model fits
the available age-aggregated data well (Appendix~\ref{sr.cal}),
and the qualitative interpretation of our findings
is unlikely to be changed by adding age stratification.
% SS: Though we have observed in single and pooled analyses lower ART coverage among younger women/FSW
% JK: I think this kind of fits with the generalzation of our findings idea:
%     that any intersection of HIV risk and barriers to ART care
%     can undermine the expected prevention impacts
Third, while we did not model pre-exposure prophylaxis (PrEP),
out analyses focus on the time period
prior to widespread PrEP availability in Eswatini.
Finally, we did not consider transmitted drug resistance.
% SS: Or differences in acquired drug resistance which may alter viral suppression rates across populations
% JK: I think next sentence kind of addresses this point?
Drug resistance is more likely to emerge
in the context of barriers to viral suppression \cite{Pham2014};
thus, cascade gaps among those at higher risk
would likely accelerate emergence of transmitted drug resistance,
and thereby amplify impacts of such gaps.
\par
% --------------------------------------------------------------------------------------------------
% JK: TODO - revise conclusion
In conclusion, HIV prevention efforts should be rooted in  
context-specific understandings of prevention gaps.
% SM: did we talk about prevention gaps before?
In the ``treatment as prevention'' era, prevention gaps include cascade gaps.
% SM: did we mention treatment as prevention before? feels like new terminologies being introduced here..?
Thus, differential cascades within and between populations at higher risk of HIV
must be monitored, quantified, modelled, and ultimately addressed
to fully realize the anticipated prevention impacts of ART.
% LW: anticipated -> optimal?
% SB: I think could finish stronger here talking about the importance of equity-lens,
%     addressing inequities to optimize treatment at an individual level and a population level.
%     ie, that prioritizing historically marginalized folks is absolutely central to doing better for everyone.
\enlargethispage{2ex} % TEMP
