\section{Discussion}\label{art.disc}
% SM: this is an excellent start, and I think almost all content there;
%     and I really like your structuring too! I think would benefit from some more work together
%     (very sorry again about my delay with this!!) re: a bit more re: concrete examples (mostly 2 paragraphs)
%     lets chat Jan 8, and before that- see what you think re: my comments/suggestions.
%     I think with one more round of review/edits with me re: the discussion section and we'll be ready to submit!
% SM: this is geting close! lets also chat about it on thursday. I'm going to ask if we can print out
%     and cut each paragraph and move 1-2 pararaphs around (just to see frst and then revert or keep the flip).
% JK: have reordered based on our chat!
We sought to explore how inequalities in the ART cascade
that intersect with HIV risk heterogeneity
may influence the model-estimated prevention impacts of ART.
% LW: Consistency again: ART coverage vs. cascade vs...
% JK: thanks!
In our applied analysis of Eswatini, we found that
slower cascade scale-up that left behind female sex workers (FSW) and their clients
could have resulted in 50--167\% more infections by 2030 \vs slower scale-up alone.
% JK: Re. removed quoted % more infections from results:
%     although I appreciate trying to be precise, I feel our analysis is more illustrative
%     because the cascades examined in counterfactual scenarios are a bit arbitrary (and not re-stated here).
%     So, I'd rather make a more general statement?
% SM: but seems obious a result then no? I felt like the strength of hte work is that it is not illustrative
%     - that we calibrated to what happend, then the imagined what could have happened instead.
%     when we do that projecting forward, we don't call it illustrave no?
% JK: Fair point! I have added back and also flipped the comparison
%     from ~50% reduction for not-left-behind vs left-behind,
%     to ~100% increase for left-behind vs not-left-behind,
%     since this seems to flow better --- what do you think?
We also found that the impact of
% SM: Prevention or transmission? I think critical to be consistent with language.
% JK: Usually we say prevention impact of ART, since ART is preventative,
%     but here it is the impact of leaving behind {subpopulations},
%     so it feels funny to read prevention impact of leaving behind ...
%     I know we discussed leaving 'transmission impact' vs just 'impact' here,
%     but I personally feel the term 'transmission impact' is not really defined so could be confusing.
%     Given the 2 prior sentences, I feel it is clear we are talking about impact re. infections?
leaving behind FSW and/or clients was largely determined by
characteristics of the client population, including:
population size, turnover, and relative HIV prevalence.
% SM: Nice opening/summary para to discussion, but I still feel like its obvious
%     and could be strengthened by saying that if the cascade had been *this bad*,
%     then could have led to X-X more infections.
%     In the several tPAF papers by MC/romain/Peter, the 'past estimates' of condom-use are like that ...
%     i.e. condom-use led to X (b/c the counterfactual was no condom use - which is unrealistic).
%     we can chat more and am ok if you want to keep it general but just seems to beg
%     why use an emblematic setting with data then vs. a purely illustrative model? :)
%     I value not trying to over-sell, but here I feel like
%     when we went to the trouble of beign data-heavy re: the model
%     but could have used a toy model to say same thing? :)
% JK: As noted above, have added back & agree now it does feel stronger to quote the numbers!
% LW: I feel the current Discussion lack contents talking about the implications of the Obj 2 findings:
%     E.g., impliations on collecting and estimating data on
%     client population size; client HIV prevalence estimates etc.
% JK: Good point --- I've added below in the para contextualizing wrt modelling literature
\par
% SM: this para could benefit from a bit more substance as it is too general for a paper that is focused on Eswatini.
%     Tell reader more about Eswatini program here, etc. - esp for journal like Lancet HIV
Eswatini has recently surpassed 95-95-95 for the population overall \cite{SHIMS3}.
The \emph{MaxART} program \cite{MaxART2} helped achieve these targets through
numerous initiatives coordinated across sectors.
Diverse stakeholders, including people living with HIV, healthcare providers,
traditional and religious leaders, community groups, and researchers
were engaged via multiple channels, including a Community Advisory Board
and specific meetings for prioritized groups (men and adolescents).
Drawing on this engagement and
social science research to understand barriers to care,
cascade services were comprehensively strengthened over 2011--2018 via investments in
training, infrastructure, anti-stigma communication, demand creation, and monitoring.
% These efforts coincide with rapid reductions in HIV incidence in recent years
% \cite{SHIMS1,SHIMS2,SHIMS3} (Figure~\ref{fig:fit.inc}).
\par
Among Swati FSW living with HIV, self-reported proportions diagnosed and on ART
were estimated to be 88\% and 86\% (98\% among diagnosed) in 2021 \cite{EswIBBS2022},
% SM: give the latest empirical estimates here...
% JK: done
though data on viral suppression were lacking.
% LW: but u mention data on VS are lacking?
% SB: Wasn't this based on self-reported treatment? I think really critical that this be specifically said here
%     ie, that treatment appears good but there are large disconnects between self-reported ART and viral suppression
%     for a number of reasons including social desirability, resistance, etc.
% JK: done
% SS: Again noting, however, that we've proposed higher VS attainment than for first two steps
%     but here we seem to suggest that potentially the opposite may be true
% JK: We did allow a higher treatment failure/discontinuation rate for FSW,
%     in addition to potentially lower VS due to turnover effects,
%     so the modelled VS could be lower (Figure~\ref{fig:fit.cascade}) than other subpopulations
Although lower than 95-95-95, these achievements are impressive
considering high HIV incidence and turnover among FSW,
which necessitate higher rates of HIV testing, ART initiation, and retention.
Indeed, alongside population-level initiatives,
key populations programs in Eswatini have continued to grow in recent years \cite{EswIBBS2022}.
Such programs offer safe access to tailored services via
community centers, mobile outreach, venue-based, and one-on-one options.
Parallel efforts have also helped reduce structural vulnerabilities,
including experiences of harassment, violence, and fear of seeking healthcare,
through community empowerment, legal supports, and
training for police and healthcare workers \cite{EswIBBS2022}.
% SM: talk about the FSW program re: testing, ART scale up, etc. here.
%     I would be specific here and not general re: inspiration, etc. (that reads a bit editorializiing perhaps on our part)
%     but rather focus on explaining how high ART coverage and VLS was achieved among FSW.
% JK: drafted above (awaiting co-author feedback)
% JK: have removed text referenced in comments below
% JK: might remove the text below depending on how it links with the inputs above.
% SM: worse in men yes? would say that
% LW: familiar?
% JK: edited for clarity.
% SM: not clear what 'survey coverage' means. also, its not clear what "familiar" means to general reader
%     specifiy / tell reader that this is a pattern seen across SSA?
% JK: edited to clarify: 'survey participation'
% SM: a bit too vague/generic statement - can we say something about clients here?
% SM: whose drivers?
% SB: Feels like getting in the weeds pretty quickly here.
%     I would lay out overall messages and then can get into specifics.
\par
% SM: this seems too obvious/vague a statement. We already say this introduction too.
%     Would cut and let discussion be focused on the implications with depth
%     as the rest of the paragraph gets into a bit.
% JK: ^ removed statements re. individual-level ART benefits
As Eswatini has shown,
efforts to maximize ART cascade coverage must not overlook marginalized populations,
% SS: "not be as routinely reached"
% JK: changed 'populations that may be harder to reach' -> 'marginalized populations'
for whom barriers to engagement in HIV care often intersect with drivers of HIV risk,
including economic insecurity, mobility, stigma, discrimination, and criminalization
\cite{Wanyenze2016,Schwartz2017,Schmidt-Sane2022,Camlin2019,Baral2019}.
Such populations can be reached effectively through
tailored services to meet their specific needs \cite{EswIBBS2022,Ehrenkranz2019},
services which can be designed and refined with ongoing community leadership and engagement.
% SB: And may want to say community-leadership rather than just engagement.
% JK: great point!
As we have shown, an equity-focused approach to ART cascade scale-up
can maximize prevention impacts, and accelerate the end of the HIV epidemic.
\par
Our study highlights the importance of reaching both FSW and their clients,
echoing recent modelling studies of South Africa and Cameroon \cite{Stone2021,Silhol2024}.
% SM: suggest make this a separate paragraph focused on clients only,
%     and would bring this below the paragraph on inequalities.
% JK: done --- though I did add back some mentions of FSW because,
%     while FSW data & programs might be strong in Eswatini, the same may not be true elsewhere,
%     and it felt a bit strange to focus on only clients,
%     when FSW still face many similar data gaps and challenges
These studies found that gaps in HIV prevention and treatment for clients
were among the largest contributors to onward transmission in recent years.
Such findings reiterate the need for improved data on both FSW and clients,
including estimates of population size, sexual behaviour, and ART cascade attainment.
These estimates may be difficult to obtain
because individuals are unlikely to report buying or selling sex in population-level surveys
due to stigma and criminalization \cite{Behanzin2013} and
because many clients are highly mobile
(including transient seasonal/occupational migration) \cite{Camlin2019}.
% SM: the term 'mobility' could mean differnet things I think even if lancet HIV audience...
% JK: rephrased but still defined a bit
% SM: was hard to glean what we want to say I thought. try something like this...?
%     Its ok if more words, better to try to write out what we want to say and then shorten it :)
% JK: have revised trying to keep the same points but shortening words.
Thus, innovative study designs, bias adjustments, and services may be needed
to understand and meet clients' needs.
% SM: feels odd to read here - does this belong in introduction instead?
% SM: specify / be more precicse re; modeling studies. b/c we talk about differnet studies in the discussion...
% LW: Would use more generic term than "mediate"
% SM: this all seems like it belongs in introduction.
% LW: Consistency in terminology: cascade gaps vs. cascade disparities
% JK: done
% SM: already said in introduction?
% JK: removed the text referenced above
\par
% SM: suggest revising this paragraph a lot. most of this seems like introduction material, and rather
%     perhaps here the message is re: demonstrating the importance of X,
%     confirming prior findings but adding new model-based evidence -- value of doing this retrospectively, etc.?
%     also, could say how this relate to prior modeling literature on prevention gaps
%     at the intersection of heightended risks in general, beyond FSW, etc. to consider clients, MSM,
%     ... i.e. as a general principle
% JK: Have revised to focus on contextualizing with other modelling
%     & highlighting the importance of clients --- what do you think?
%     Only thing I've omitted from suggested is the 'value of doing this retrospectively',
%     since this is noted below in the 'strengths' para
While numerous modelling studies have examined
the potential prevention impacts of ART cascade scale-up
\cite{Knight2022sr} (Appendix~\ref{sr.ric}),
our study is the first to explore the impact of
inequalities in ART cascade across subpopulations
with consistent population overall cascade across scenarios.
% SM: lets be more precise - what did we add / confirm re: prior studies?
%     its not clear from this sentence how our findings differ or relate to that of MMG's paper
% JK: Have clarified how ours is unique, re. consistent population overall cascade across scenarios
%     and have now removed the text referenced in the comments below.
% SM: doesn't flow yet re: nevertheless and am not able to easily see the links/flow.
%     among the first (or even the first)...and then this sentence. is disjointed.
%     let me know if helpful to edit/revise or if you want to have a go first?
%     I will need to read MM-G's paper again or if you can recap what they did/found
%     so we can more clearly relate our findigns to this wider literature?
% JK: have revised to try and improve flow -- what do you think?
Similar work by \citet{Marukutira2020} illustrated the limited impact of
achieving 95-95-95 for only citizens and not immigrants in Botswana,
while \citet{Maheu-Giroux2019cost} illustrated the high cost-effectiveness of
prioritizing key populations (including clients) for ART in C\^{o}te d’Ivoire.
% JK: not sure whether the above details re. populations explored in 2 studies are needed
%     the phrasing is a bit awkward, and maybe not needed.
% SM I think we need to edit above - was not clear what we want to say
% JK: have edited to spell out a bit more the findings of each paper.
Indeed, our findings are likely generalizable to other epidemic contexts.
% SM: other or similar epidemic contexts?
% JK: I am comfortable saying 'other', not just 'similar'
%     but it might be stretching too far without showing any evidence
HIV prevalence ratios between key populations and the population overall
are relatively low in Eswatini \vs elsewhere;
thus, the impact of cascade inequalities among key populations in other contexts
would likely be even greater than we found for Eswatini.
Moreover, as HIV incidence declines in many settings,
transmissions may become concentrated among key populations \cite{Brown2019},
further magnifying the impact of cascade inequalities.
% SM: clarify --> does this mean that even if equalities --> woudl still become concentrated
%     b/c prevention gaps still exist even if cascade is equal?
% LW: Saw Stef's edits using wording like inequity; just to make sure consistency
% JK: thanks -- got it!
\par
A primary strength of our analysis is the use of
observed ART cascade scale-up to 95-95-95 in Eswatini as the base case,
with plausible cascade inequalities explored in counterfactual scenarios.
As noted above, the available data suggest that Eswatini has
minimized cascade inequalities which persist elsewhere \cite{Hakim2018}.
Thus, our counterfactual scenarios directly estimate
the consequences of failing to address these inequalities.
% JK: rephrased SM edits for brevity
%     hopefully the 'failing to address' is not too negative?
% LW: Elaborate why this is a strength in relation to ur research question
% JK: I think it is a bit self-evident?
Second, drawing on our conceptual framework for risk heterogeneity \cite[Table~1]{Knight2022sr}
and multiple sources of context-specific data
\cite{SDHS2006,SHIMS1,SHIMS2,Baral2014,EswKP2014,EswIBBS2022},
we captured several dimensions of risk heterogeneity, including:
heterosexual anal sex,
% LW: Was anal sex explicitly mentioned in methods. If not - please add
% JK: Yes, but only in Table~\ref{tab:art.het} (1)
four types of sexual partnerships,
sub-stratification of FSW and clients into higher and lower risk strata,
and subpopulation turnover
(Table~\ref{tab:art.het}, Appendix~\ref{mod}).
% SM: say why this is imporatnt/good/makes for more robsut analysis...
%     not all readers are modellers. (not super self-evident why this is a good thing per se)
% JK: done, below
Accurate modelling of risk heterogeneity
has been shown to mediate model-estimated ART prevention impacts \cite{Hontelez2013},
and is especially important when considering differential ART scale-up across subpopulations.
Finally, our analytic approach to Objective \ref{obj:art.2},
in which epidemic conditions are conceptualized as potential effect modifiers
% and regression models are employed to examine influence of key parameters on projected population-level impact
% JK: I think to remove the suggested above, b/c
%     prior approaches to sensitivity analysis (PRCC) essentially use linear models
%     just in a causal framework to try to isolate effects
represents a unique methodological contribution to the HIV modelling literature.
\par
% LW: mention did not PrEP
% JK: done (below)
Our study also has limitations.
% SM: an editor once suggested not to number at the start, but just number in the text...
% JK: okay!
% SM: for each limitation (except re: MSM), tell reader
%     how this could change/bias (or not) the findings
%     see the inequalities paper by LW as an example
%     i learned a lot from how she does the limitations sections :)
% JK: I have tried for PrEP and TDR, but I am a bit wary of speculating
%     about influence of age, MSM, PWID, etc. --- more details in (1) below
First, we did not model pre-exposure prophylaxis (PrEP).
% SM: can we say this in methods instead? and instead
%     have something that comes about our generalizble statement,
%     re: future work would include... how inequalities in context of
%     PrEP scale-up, drug resistance, aging epidemic, etc.
However, our analyses focus on the time period
prior to widespread PrEP availability in Eswatini.
Second, we did not consider transmitted drug resistance (TDR).
% SS: Or differences in acquired drug resistance
%     which may alter viral suppression rates across populations
% JK: I think next sentence kind of addresses this point?
However, drug resistance is more likely to emerge
in the context of barriers to viral suppression;
thus, lower cascade among those at higher risk
would likely accelerate emergence of transmitted drug resistance,
and thereby magnify our findings.
Finally, our model structure did not include age,
and we only considered heterosexual HIV transmission in Eswatini.
% SM: I think ther are 2 issues here:
%     (1) how does excluding MSM, excluding age change the actual projected impact of inequalities?
%     (2) generalizable re: implications across popualtions/heterogenetiy.
%     So would include statement about did not include trasmission via
%     injecting drugs, nor sex between men, and did not stratify by age.
%     This could mean that our estimates of the impact of inequalities are...? then say...
%     however, the implications of our findings surrounding inequalities would generalize...
%     (and end limitations paragraph with this).
% JK: I think the implications for (2) are noted below already?
%     For (1), I am a bit wary of speculating about these without modelling,
%     as I'm not confident I could predict the influence.
%     Plus, e.g. if we added MSM, the % transmission to/from clients may change,
%     but our research question would have likely considered MSM too in cf scenarios,
%     so I think it's hard to make a general statement about how
%     in/excluding MSM would influence the 'impact of inequalities'
%     i.e. we would need to specify inequalities for whom
% RK: A sentence or two about barriers to care and ART rollout among MSM in SSA would fit well, if space allows
% JK: Although did not call out MSM specifically
%     I've tried to add more specific barriers outlined above
%     (and focused a bit on clients due to our focus and findings)
Future work can explore adaptation of the model to consider
PrEP, TDR, age stratification, and additional modes of HIV transmission.
While the magnitude of our results may change with such adaptations,
we do not expect that the qualitative interpretation would change.
% SM: re-frame as the direction of influence is expected to remain the same, but the magnitude ...
%     [and thus, the overall insights surrounding the influence of inequalities
%     is not expected to be influenced by age-stratification]...
In fact, our findings would likely generalize
to other transmission networks and determinants of risk heterogeneity,
including other key populations and subpopulations such as
highly mobile populations and young women \cite{Camlin2019,Cheuk2020}.
% SM: I know often used, but the term "mobile populations" could mean different things to differnet audience
%     (for healthcare workers, it means those who can walk and who cannot independtly walk) :)
% JK: I think given the context of the sentence and references it should be clear enough?
%     Can revise if reviewers request?
% SS: Though we have observed in single and pooled analyses lower ART coverage among younger women/FSW
% JK: I think this kind of fits with the generalzation of our findings idea:
%     that any intersection of HIV risk and barriers to ART care
%     can undermine the expected prevention impacts
% SM: agree - lets end the limitations paragraph with this (see comment above).
% JK: done.
\par
In conclusion, the HIV response must remain rooted in
context-specific understandings of inequalities in HIV risk and in access to HIV services,
which often stem from common upstream factors.
% SM: did we talk about prevention gaps before?
% SM: did we mention treatment as prevention before? feels like new terminologies being introduced here..?
% JK: revised above phrasing completely
Thus, differences in ART cascade within and between subpopulations at higher risk of HIV
must be monitored, characterized, and addressed
to fully realize the anticipated benefits of ART
at both the individual and population levels.
% LW: anticipated -> optimal?
% SB: I think could finish stronger here talking about the importance of equity-lens,
%     addressing inequities to optimize treatment at an individual level and a population level.
%     ie, that prioritizing historically marginalized folks is absolutely central to doing better for everyone.
% JK: have revised to highlight equity lens
\enlargethispage{2ex} % TEMP
