\newcommand{\pu}[2]{#1\,+\,#2}
\begin{ric}
  \paragraph{Evidence before this study}
  There has been extensive interest in modelling the potential prevention benefits of
  achieving the UNAIDS 90-90-90 / 95-95-95 goals \cite{909090,959595},
  but little attention has been paid to \emph{who} comprise the ``10-10-10''.
  We adapted and updated our previous scoping review of transmission modelling studies
  applied to assess the prevention impacts of HIV treatment scale-up in Sub-Saharan Africa (SSA)
  \cite{Knight2022sr} (full details in Appendix~\ref{sr.ric}).
  Our search for concepts: SSA, modelling, HIV, and cascade of care
  yielded (\pu{prior}{update}) \pu{1367}{373} unique studies, of which:
  \pu{7}{5} explicitly examined the prevention impacts of achieving 90-90-90 or greater.
  Several of these studies (Table~\ref{tab:ric.refs}) considered differences in
  status quo rates of diagnosis, treatment initiation, and/or treatment failure/discontinuation
  --- mainly by sex, age, and occasionally risk.
  However, only 3 studies considered scenarios in which
  sub-populations were ``left behind'' during scale-up towards 90-90-90+:
  key populations in \cite{Maheu-Giroux2017art},
  men in \cite{Reidy2019}, and
  migrants in \cite{Marukutira2020}.
  Since the modelled epidemic in \cite{Maheu-Giroux2017art} (C\^{o}te d'Ivoire)
  is relatively concentrated among key populations,
  the potential impact of ``leaving behind'' key populations
  in a high-prevalence epidemic such as Eswatini has not been examined.
  \paragraph{Added value of this study}
  % TODO
  \paragraph{Implications of all the available evidence}
  % TODO
\end{ric}
