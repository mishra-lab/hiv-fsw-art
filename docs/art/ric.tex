\newcommand{\pu}[2]{#1\,+\,#2}
\begin{ric}
  \paragraph{Evidence before this study}
  We adapted and updated a scoping review of HIV transmission modelling studies examining
  the prevention impacts of antiretroviral therapy (ART) cascade scale-up in Sub-Saharan Africa
  (full details in Appendix~\ref{sr.ric}).
  Our search yielded (\pu{prior}{update}) \pu{1367}{373} unique studies, of which:
  \pu{7}{5} examined the prevention impacts of achieving
  the UNAIDS 90-90-90 goals or greater (Table~\ref{tab:ric.refs}).
  Some of these studies considered differences in baseline rates of
  diagnosis, treatment initiation, and/or treatment failure/discontinuation
  --- mainly by sex, age, and occasionally risk.
  Three studies specifically examined scenarios of unequal cascade attainment across subpopulations,
  predicting substantially more infections when subpopulations at higher risk were left behind.
  However, these studies did not maintain consistent population overall attainment across scenarios,
  and considered only hypothetical scenarios.
  \paragraph{Added value of this study}
  We develop and calibrate a detailed model of heterosexual HIV transmission
  and observed ART cascade scale-up in Eswatini,
  drawing on population-level and female sex worker (FSW)-specific surveys.
  We show that even for high overall cascade attainment,
  cascade inequalities across subpopulations experiencing differential HIV risk
  --- namely lower cascade among FSW and their clients ---
  can undermine the prevention impacts of ART.
  \paragraph{Implications of all the available evidence}
  The individual-level and partnership-level benefits of ART are clear and important.
  Population-level prevention benefits also have the potential to help rapidly end the HIV epidemic.
  However, these benefits will not be fully realized in any epidemic context
  unless an equity-focused approach ensures no subpopulations are left behind ---
  especially those experiencing intersecting drivers of HIV risk and barriers to care,
  including but not limited to FSW and their clients.
\end{ric}
